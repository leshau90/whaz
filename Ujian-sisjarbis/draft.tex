\documentclass[12pt]{article}

\usepackage[a4paper,left=2.5cm,right=2.5cm,top=2.5cm,bottom=2.5cm]{geometry}
% \usepackage[bahasa]{babel}
\usepackage[style=apa]{biblatex}
\addbibresource{main.bib}

\usepackage{titlesec}

\titleformat{\section}
  {\normalfont\fontsize{13}{15}\bfseries}{\thesection}{1em}{}

\renewcommand\thesection{\Alph{section}.}
% \renewcommand\thesubsection{\thesection.\Alph{subsection}}

\usepackage{setspace} \onehalfspacing
\usepackage{lipsum}
\usepackage{graphicx}
\usepackage{hyperref}
% \usepackage{biblatex}
% \addbibresource{main.bib}
% \usepackage{apacite}
\usepackage{longtable}
% \usepackage[backend=biber,style=apa,citestyle=apa,sorting=ynt]{biblatex}
\usepackage{usebib} 
\bibinput{main}
\usepackage{indentfirst}
\graphicspath{ {./images/} }
\title{TAKE HOME TEST }


\begin{document}
\thispagestyle{empty}
\begin{center}
    \textbf{SISTEM INFORMASI BISNIS-SISTEM INFORMASI BISNIS }\\
    Take Home Test\\
    \vspace*{11\baselineskip}
    \includegraphics[width=0.4\textwidth,height=0.4\textwidth]{glogo}  \\
    \vspace*{11\baselineskip}
    disusun oleh:\\
    \textbf{Ilman Samhabib (91122010)}\\
    \textbf{Universitas Gunadarma}\\
    \textbf{2023}\\
\end{center}
\newpage


\subsection*{Soal 1}
\noindent


Apakah yang dimaksud deengan Valas (foreign exchange rate)? Mengapa beberapa mata uang asing masuk dalam kategori hard currency? Jelaskan!\\

Valas, adalah harga atau nilai yang digunakan untuk menukar atau memberi nilai untuk mata uang tertentu yang didasarkan pada mata uang lainnya. \emph{Hard currencies} adalah mata uang - mata uang yang dikeluarkan atau dipakai pada negara yang stabil kondisi ekoonomi dan poltiknya. Sehingga menurut pelaku pasar valas atau pun pihak yang berkepentingan lainnya, mata uang tersebut,akan stabil (tidak tiba-tiba naik ataupun turun tajam nilainya untuk ditukarkan), dan karena kestabilan internal negaranya mata uang ini sangat liquid ataupun dapat diuangkan/ditukar setiap saat jika ada keperluan dengan nilai yang tidak berfluktuasi secara tajam. Karena hal ini pula Mata uang - mata uang ini juga dijadikan sebagai patokan untuk menilai kekayaan tertentu yang secara multinasional \autocite{onlineshayla2022}

    Beberapa Mata uang menjadi \emph{hard currency} karena hal-hal \autocite{onlineshayla2022} berikut:

\begin{enumerate}
    \item \textbf{Stabilitas Politik dan Ekonomi:} \emph{Hard Currency} berasal dari negara-negara dengan ekonomi dan sistem politik yang stabil, menjadikannya dapat diandalkan dan diterima secara luas di seluruh dunia sebagai bentuk pembayaran atas barang dan jasa.

    \item \textbf{Likuiditas dan Penyimpanan Kekayaan:} \emph{Hard Currency} berperan sebagai penyimpanan kekayaan yang likuid dan sebagai tempat aman ketika mata uang domestik mengalami kesulitan, menjadikannya berharga dan lebih diutamakan dibanding mata uang domestik.

    \item \textbf{Kekuatan Global dan Stabilitas:} \emph{Hard Currency} berasal dari negara-negara yang merupakan kekuatan global dan secara umum stabil secara politik, yang menambah kestabilan dan kehandalan mereka.

\end{enumerate}

Kelemahan Mata Uang Keras --
Mata uang keras memiliki nilai lebih tinggi dibanding mata uang lainnya. Sebagai contoh, pada 6 November 2020, pasar valuta asing (FX) diperdagangkan dengan kurs 6,61 yuan per dolar Amerika Serikat dan 73,97 rupee per dolar. Kurs-kurs ini merugikan bagi importir China dan India, namun menguntungkan untuk neraca transaksi berjalan. Kurs yang lemah membantu para eksportir sebuah negara karena membuat ekspor lebih kompetitif (atau lebih murah) di pasar komoditas dan pasar internasional lainnya. Dalam beberapa tahun terakhir, Tiongkok telah dihadapkan pada tuduhan memanipulasi kursnya untuk menekan harga dan merebut pangsa pasar internasional.

Hal ini dapat merugikan negara-negara yang mengimpor barang dari China dan India karena nilai tukar yang rendah membuat harga impor menjadi lebih tinggi. Di sisi lain, kebijakan nilai tukar yang lemah dapat membantu negara-negara tersebut mencapai saldo transaksi berjalan yang lebih baik karena membuat ekspor lebih menguntungkan. Namun, kebijakan semacam itu sering kali mendapat kritik karena dianggap sebagai upaya manipulasi untuk keuntungan ekonomi nasional \autocite{chenCurrency2023}.

Kelemahan lainnya dari mata uang keras adalah potensi ketidaksetaraan dalam perdagangan internasional, di mana negara-negara dengan mata uang lemah mungkin menghadapi kesulitan bersaing di pasar global. Selain itu, tuduhan manipulasi nilai tukar dapat menciptakan ketegangan perdagangan antara negara-negara dan memicu sengketa perdagangan internasional.

\subsection*{Soal 2}
\noindent
Apakah yang dimaksud dengan perdagangan internasional? Mengapa perdagangan internasional penting?\\

Perdagangan internasional adalah kegiatan jual beli barang atau jasa antara satu negara dengan negara lainnya\autocite{yogasukmana2023}. Perdagangan internasional penting karena memiliki beberapa manfaat, di antaranya:

\begin{enumerate}
    \item \textbf{Meningkatkan Pertumbuhan Ekonomi:} Perdagangan internasional dapat membawa pertumbuhan ekonomi dalam negeri, baik secara langsung maupun tidak langsung. Pertumbuhan ekonomi dapat terjadi melalui pengaruh yang ditimbulkan terhadap alokasi sumber daya dan efisiensi\autocite{aamSlamet2021}.

    \item \textbf{Membuka Peluang Kerja:} Perdagangan internasional membantu menghasilkan lebih banyak lapangan pekerjaan melalui pembangunan industri-industri baru guna memenuhi permintaan produk di berbagai negara. Kondisi tersebut tentu akan membantu negara-negara menurunkan tingkat pengangguran\autocite{nugroho2021}.

    \item \textbf{Memperluas Pasar:} Perdagangan internasional dapat menambah pasar bagi perusahaan. Dengan adanya perdagangan internasional, pengusaha bisa menjalankan mesin-mesin produksinya secara maksimal dan menjual kelebihan produk yang dihasilkan ke luar negeri \autocite{nugroho2021}.

    \item \textbf{Meningkatkan Pendapatan:} Perdagangan internasional dapat meningkatkan pendapatan negara. Tingginya produktivitas akan meningkatkan pendapatan \autocite{nugroho2021}.

    \item \textbf{Meningkatkan Kualitas Hidup:} Perdagangan internasional dapat meningkatkan kualitas hidup masyarakat. Dengan adanya perdagangan internasional, masyarakat dapat memperoleh barang dan jasa yang tidak tersedia di negaranya \autocite{rosyda2021}.
\end{enumerate}


Dalam perdagangan internasional, terdapat beberapa istilah yang perlu dipahami, seperti ekspor, impor, dan neraca perdagangan. Ekspor adalah kegiatan menjual barang atau jasa ke negara lain, sedangkan impor adalah kegiatan membeli barang atau jasa dari negara lain. Neraca perdagangan adalah selisih antara nilai ekspor dan impor suatu negara\autocite{Aflah2023KeterkaitanHP}.


\subsection*{Soal 3}
\noindent
3. Jelaskan pendapat saudara tentang adanya AFTA dan bagaimana manfaatnya untuk Indonesia dengan menjadi anggota AFTA tersebut!\\

AFTA (ASEAN Free Trade Area) adalah bentuk kerja sama perdagangan dan ekonomi di wilayah ASEAN, berupa kesepakatan agar tercipta situasi perdagangan yang seimbang, dengan penurunan tarif barang dagang serta pajak bagi negara-negara di Asia Tenggara. Sebagai salah satu anggota dari negara ASEAN, pembentukan AFTA telah memberikan beberapa dampak positif serta keuntungan bagi Indonesia\autocite{qothru2021}. Beberapa manfaat AFTA bagi Indonesia antara lain:
\begin{enumerate}
    
\item Peluang Ekspor: AFTA memberikan peluang bagi para pengusaha kecil dan menengah untuk melakukan ekspor barang produksinya, sehingga mampu membuka peluang mereka untuk mendapatkan pasar luar negeri\autocite{tysara2022}.
    
\item Meningkatkan Perdagangan: AFTA meningkatkan perdagangan antar negara anggota ASEAN (Intra ASEAN Trade), sehingga dapat membantu meningkatkan perekonomian Indonesia\autocite{tysara2022}.

\item Meningkatkan Daya Saing: AFTA membuat Indonesia untuk lebih bisa menghasilkan komoditas yang kompetitif di pasar ASEAN. Salah satu komoditas Indonesia yang dapat bersaing dengan negara lainya adalah komoditas pertanian, seperti kelapa sawit, karet, kakao, dan kopi yang merupakan bahan yang sangat diminati oleh negara ASEAN maupun di luar negeri\autocite{tysara2022}.

\item Meningkatkan Peluang Investasi: AFTA menarik lebih banyak investasi asing langsung (Foreign Direct Investment) ke Indonesia\autocite{Putri2022AnalisisKA}.

\item Meningkatkan Daya Saing Ekonomi: AFTA meningkatkan daya saing ekonomi negara-negara ASEAN dengan menjadikan ASEAN sebagai basis produksi pasar dunia untuk menarik investasi dan meningkatkan perdagangan antar anggota ASEAN\autocite{Hadi2016AnalisisKD}.
\end{enumerate}


Dalam kesepakatan AFTA, Indonesia berharap dapat meningkatkan perekonomian secara menyeluruh. Dengan adanya AFTA, Indonesia dapat memperluas pasar, meningkatkan perdagangan, dan meningkatkan daya saing ekonomi. Oleh karena itu, Indonesia menjadi salah satu negara yang paling awal menyetujui AFTA.

implementasi AFTA (ASEAN Free Trade Area) sebagai sebuah sistem ataupun rezim perdagangan ASEAN, dengan pendekatan liberalisme ekonomi dan integrasi ekonomi dan politik, meningkatkan daya saing dan memperkuat integrasi sekawasan. Liberalisasi ekonomi awalnya diterapkan untuk kepentingan ekonomi dan kemudian berkembang menjadi kepentingan politik, memperkukuh rezim AFTA di negara-negara ASEAN dan internasional. Hasilnya menunjukkan bahwa perdagangan bebas di ASEAN-AFTA memungkinkan aliran barang tanpa hambatan, meningkatkan aktivitas produksi di antara negara-negara ASEAN, dan memperkuat posisi ASEAN dalam rezim perdagangan internasional\autocite{gani2021AFTAMR}.

% Citations:
% [1] https://www.detik.com/edu/detikpedia/d-5844191/apa-yang-dimaksud-afta-ini-tujuan-dan-dampaknya-bagi-indonesia
% [2] https://setnasasean.id/news/read/pengertian-afta-beserta-tujuan-dan-pengaruhnya-bagi-indonesia
% [3] https://www.merdeka.com/jatim/tujuan-afta-dan-latar-belakang-pendiriannya-menarik-dipelajari-kln.html
% [4] https://www.liputan6.com/hot/read/5113176/afta-adalah-asean-free-trade-area-ini-tujuan-didirikan-anggota-dan-manfaatnya
% [5] https://www.bola.com/ragam/read/4608453/pengertian-afta-beserta-tujuan-dan-pengaruhnya-bagi-indonesia
\subsection*{Soal 4}
\noindent
4. Aktivitas apa saja yang termasuk dalam International Business? Jelaskan masing-masing!\\

Kegiatan bisnis internasional mencakup berbagai operasi lintas batas yang melibatkan transaksi, perdagangan, investasi, dan kerja sama antara perusahaan dan pemerintah di berbagai negara . Kegiatan ini sangat penting bagi ekonomi global karena memfasilitasi pertukaran barang, jasa, modal, dan teknologi secara internasional \autocite{bussinessToday2020}. Beberapa kegiatan kunci yang termasuk dalam lingkup bisnis internasional meliputi perdagangan internasional, investasi langsung asing (FDI), lisensi dan waralaba, aliansi strategis dan usaha patungan, serta sumber daya global.
\begin{enumerate}
    \item Perdagangan Internasional: Melibatkan pertukaran barang dan jasa antara negara-negara, diatur oleh perjanjian perdagangan internasional, tarif, kuota, dan kebijakan perdagangan. Perusahaan terlibat dalam perdagangan internasional untuk mengakses pasar baru, memperoleh bahan baku atau komponen dengan harga yang kompetitif, dan memanfaatkan keunggulan komparatif dalam produksi.
    \item Investasi Langsung Asing (FDI): Merujuk pada investasi yang dilakukan oleh perusahaan berbasis di satu negara ke perusahaan atau entitas di negara lain. FDI memungkinkan perusahaan untuk memperluas operasinya secara global, mengakses pangsa pasar baru, mendapatkan efisiensi biaya, dan mendapatkan keuntungan strategis melalui pendirian fasilitas atau akuisisi bisnis yang sudah ada di pasar asing.
    \item Lisensi dan Waralaba: Lisensi melibatkan memberikan izin kepada perusahaan asing untuk menggunakan kekayaan intelektual seperti paten, merek dagang, atau teknologi dengan membayar biaya atau royalti. Waralaba melibatkan pemberian hak untuk menggunakan model bisnis, merek, dan proses operasional perusahaan dengan imbalan biaya dan royalti. Kesepakatan ini memungkinkan perusahaan untuk memperluas kehadirannya secara internasional tanpa melakukan investasi substansial dalam aset fisik.
    \item Aliansi Strategis dan Usaha Patungan: Perusahaan membentuk aliansi strategis atau usaha patungan dengan mitra asing untuk berkolaborasi dalam proyek-proyek tertentu atau memasuki pasar baru bersama-sama. Kemitraan ini memungkinkan perusahaan untuk berbagi sumber daya, keahlian, risiko, dan pengetahuan pasar sambil memanfaatkan kekuatan masing-masing untuk mencapai manfaat bersama.
    \item Sumber Daya Global (\emph{global sourcing}): Melibatkan pengadaan barang, jasa, atau komponen dari pemasok yang berlokasi di berbagai negara. Perusahaan terlibat dalam sumber daya global untuk mengakses input dengan biaya efektif, mendiversifikasi rantai pasokan, mengurangi risiko yang terkait dengan lokasi satu-satunya sumber, dan memanfaatkan kemampuan khusus yang tersedia di berbagai wilayah.

\end{enumerate}



% Hill, C. W. L., Hult, G. T. M., & Wickramasekera, R. (2019). Global Business Today (11th ed.). McGraw-Hill Education. (Print)

% Daniels, J. D., Radebaugh, L. H., & Sullivan, D. P. (2018). International Business: Environments and Operations (16th ed.). Pearson. (Print)

% Rugman, A., & Collinson, S. (2019). International Business (7th ed.). Pearson. (Print)

% Czinkota, M.R., Ronkainen I.A., & Moffett M.H. (2009). International Business (8th ed.). John Wiley & Sons. (Print)

% Peng M.W. (2016). Global Business (4th ed.). Cengage Learning. (Print)


\subsection*{Soal 5}
\noindent
Apa yang dimaksud dengan etika bisnis? Bagaimana kaitannya dengan tanggung jawab sosial perusahaan? Berikan contohnya!\\


Etika bisnis merujuk pada prinsip-prinsip moral dan nilai-nilai yang mengatur perilaku dan keputusan dalam konteks bisnis. Hal ini melibatkan pertimbangan terhadap keadilan, kejujuran, tanggung jawab, dan kepatuhan terhadap hukum dalam setiap tindakan yang dilakukan oleh perusahaan atau individu di dunia bisnis. Etika bisnis juga mencakup bagaimana perusahaan berinteraksi dengan karyawan, konsumen, pemasok, dan masyarakat secara umum.

Tanggung jawab sosial perusahaan (CSR) adalah konsep yang berkaitan dengan bagaimana perusahaan mempertimbangkan dampak sosial, lingkungan, dan ekonomi dari keputusan dan tindakan mereka. CSR mencakup komitmen perusahaan untuk bertindak secara etis, berkontribusi pada pembangunan berkelanjutan, serta memperhatikan kepentingan semua pemangku kepentingan yang terlibat dalam operasi perusahaan \autocite[340]{trevino2016managing}.

Kaitan antara etika bisnis dan tanggung jawab sosial perusahaan sangat erat. Etika bisnis memberikan landasan moral bagi praktik bisnis yang bertanggung jawab secara sosial. Dengan menerapkan prinsip-prinsip etika bisnis dalam operasinya, perusahaan dapat memastikan bahwa mereka tidak hanya mencari keuntungan semata, tetapi juga memperhatikan dampak sosial dan lingkungan dari kegiatan bisnis mereka \autocite{trevino2016managing}.

Salah satu contoh kaitan antara etika bisnis dan tanggung jawab sosial perusahaan adalah ketika sebuah perusahaan memutuskan untuk menggunakan bahan baku yang ramah lingkungan dalam produksinya. Dengan demikian, perusahaan tersebut tidak hanya menjaga reputasinya sebagai entitas yang bertanggung jawab secara sosial, tetapi juga memberikan kontribusi positif terhadap pelestarian lingkungan\autocite[50]{trevino2016managing}.

% Referensi:

% Ferrell, O.C., Fraedrich, J., & Ferrell, L. (2018). Business Ethics: Ethical Decision Making & Cases (12th ed.). Cengage Learning. (Print)
% Carroll, A.B., & Buchholtz, A.K. (2014). Business and Society: Ethics and Stakeholder Management (9th ed.). Cengage Learning. (Print)
% Crane, A., & Matten, D. (2016). Business Ethics: Managing Corporate Citizenship and Sustainability in the Age of Globalization (4th ed.). Oxford University Press. (Print)
% Velasquez, M.G. (2018). Business Ethics: Concepts and Cases (8th ed.). Pearson Education. (Print)
% Treviño, L.K., & Nelson, K.A. (2016). Managing Business Ethics: Straight Talk about How to Do It Right (7th ed.). Wiley. (Print)

\subsection*{Soal 6}
\noindent
Jelaskan pendapat saudara tentang adanya Uni Eropa dan penggunaan mata uang bersama EURO dalam kaitannya dengan perdagangan di kawasan itu!\\


Uni Eropa (UE) adalah sebuah organisasi politik dan ekonomi yang terdiri dari 27 negara anggota di Eropa. Salah satu aspek penting dari integrasi Eropa adalah penggunaan mata uang bersama, yaitu Euro, yang diperkenalkan pada tahun 1999 dan mulai beredar secara fisik pada tahun 2002. Uni Eropa dan penggunaan mata uang bersama Euro memiliki dampak yang signifikan dalam kaitannya dengan perdagangan di kawasan tersebut \autocite{baldwin2015economics}.
\begin{enumerate}
    \item Keberadaan Uni Eropa menciptakan pasar tunggal yang besar di mana perdagangan antar anggota UE dapat berlangsung tanpa hambatan. Hal ini memungkinkan perusahaan untuk lebih mudah menjual produk mereka ke negara-negara lain di UE tanpa harus menghadapi hambatan tarif atau non-tarif yang biasanya terjadi dalam perdagangan internasional. Dengan demikian, adopsi Euro sebagai mata uang bersama juga memfasilitasi perdagangan di antara negara-negara anggota UE dengan menghilangkan risiko fluktuasi nilai tukar antar mata uang .
    \item Penggunaan Euro juga memudahkan transaksi bisnis di antara negara-negara anggota UE. Dengan menggunakan mata uang yang sama, perusahaan  menghindari biaya konversi mata uang dan risiko fluktuasi nilai tukar yang  mempengaruhi keuntungan mereka. Hal ini membuat proses pembayaran dan transaksi bisnis menjadi lebih efisien dan transparan di seluruh kawasan UE \autocite[280]{baldwin2015economics}.
    \item Adopsi Euro juga memberikan stabilitas ekonomi bagi negara-negara anggota UE. Dengan memiliki mata uang bersama, negara-negara anggota tidak lagi bergantung pada kebijakan moneter nasional mereka sendiri. Sebagai gantinya, kebijakan moneter ditetapkan oleh Bank Sentral Eropa, yang bertanggung jawab atas menjaga stabilitas harga di seluruh kawasan Euro. Hal ini menciptakan lingkungan ekonomi yang lebih stabil dan dapat meningkatkan kepercayaan investor serta konsumen dalam perdagangan di kawasan tersebut.
    \item Penggunaan Euro juga memperkuat posisi Uni Eropa dalam perdagangan internasional. Sebagai salah satu mata uang utama di dunia, Euro memberikan UE kekuatan tawar yang lebih besar dalam negosiasi perdagangan dengan mitra dagang di luar kawasan tersebut. Selain itu, Euro juga menjadi alternatif yang menarik bagi para pelaku bisnis internasional dalam melakukan diversifikasi risiko mata uang \autocite[352]{baldwin2015economics}.
    \item Adopsi Euro juga mendorong integrasi ekonomi di antara negara-negara anggota UE. Dengan menggunakan mata uang bersama, negara-negara anggota menjadi lebih terikat secara ekonomi dan hal ini mendorong kerjasama lebih   lanjut dalam berbagai bidang seperti investasi, infrastruktur, dan inovasi \autocite[40]{baldwin2015economics}.


\end{enumerate}


 Uni Eropa dan penggunaan mata uang bersama Euro telah memberikan manfaat besar dalam memfasilitasi perdagangan di kawasan tersebut melalui penciptaan pasar tunggal yang besar, efisiensi transaksi bisnis, stabilitas ekonomi, kekuatan tawar dalam perdagangan internasional, serta integrasi ekonomi di antara negara-negara anggota.

% References:

% Baldwin, R., & Wyplosz, C. (2015). The Economics of European Integration (Print).
% De Grauwe, P., & Mongelli, F.P. (2008). New Directions for European Monetary Union: Implications for Economic Policy and Financial Markets (Print).
% Bayoumi, T., & Eichengreen, B. (1997). Ever closer to heaven? An optimum-currency-area index for European countries (Print).
% European Central Bank. (n.d.). The euro (Web).
% European Union. (n.d.). The single market (Web).

% File jawaban mohon diketik dengan MS Word, dan dikumpulkan secara kolektif melalui ketua kelas, dikirim ke medyawati70@gmail.com atau diunggah pada Google drive kelas, paling lambat tanggal 14 November 2023 jam 21.00


\printbibliography[title=Daftar Pustaka]

\end{document}