\documentclass{article}

\usepackage[a4paper,left=2.5cm,right=2.5cm,top=2.5cm,bottom=2.5cm]{geometry}
\usepackage[bahasa]{babel}

\usepackage{lipsum}
\usepackage{graphicx}
\usepackage{hyperref}
% \usepackage{biblatex}
% \addbibresource{main.bib}
\usepackage{apacite}
\usepackage{longtable}
% \usepackage[backend=biber,style=apa,citestyle=apa,sorting=ynt]{biblatex}
% \addbibresource{main.bib}
\usepackage{usebib}
\bibinput{main}
\graphicspath{ {./images/} }
\title{Tugas SPK}
\author{Ilman Samhabib 91122010\\62/MMSI/SIB}
\date{\today}

\begin{document}

% \maketitle
% \titlepage
% \newpage
% \tableofcontents
% \newpage
\begin{center}
    Tugas SPK
    \\ Tema: \textbf{Programmatic Advertising} 
    \\ 991122010 62/MMSI/SIB
    \\ Ilman Samhabib
\end{center}

\subsubsection*{Nomor 1}
Deskripsikan jumlah atribut/field yang anda gunakan pada dataset. Beri penjelasan pada masing masing dataset
\bigbreak
Data set yang telah dimerge menggunakan dataset olist Brazilian E-commerce \cite{olist_2018}  yang terdiri dari beberapa file csv.
\subsubsection*{Nomor 2}
Pola apa yang bisa anda angkat sebagai masalah penelitian dari dataset yang anda dapatkan
\subsubsection*{Nomor 3}
Berdasarkan beberapa literature review yang anda baca, metode apa yang anda gunakan dalam menyelesaikan maslah pada dataset. Apa alasan menggunakan metode tersebut
\subsubsection*{Nomor 4}
Buat simulasinya dengan menggunakan tools datamining 
\subsubsection*{Nomor 5}
Apa yang bisa anda simpulkan dari kesuluruha proses yang anda lakukan menggunakan dataset dan metode yang anda purpose
too\cite{ECommerceWebsWijaya2021}



\bibliographystyle{apacite}
\bibliography{main}
% \printbibliography


\end{document}