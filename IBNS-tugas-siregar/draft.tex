\documentclass[12pt]{article}

\usepackage[a4paper,left=1cm,right=1cm,top=1cm,bottom=1.5cm]{geometry}
% \usepackage[bahasa]{babel}
\usepackage[style=apa]{biblatex}
\addbibresource{main.bib}

\usepackage{titlesec}

\titleformat{\section}
  {\normalfont\fontsize{13}{15}\bfseries}{\thesection}{1em}{}

\renewcommand\thesection{\Alph{section}.}
% \renewcommand\thesubsection{\thesection.\Alph{subsection}}
\usepackage{lscape}
\usepackage{setspace} \onehalfspacing
\usepackage{lipsum}
\usepackage{graphicx}
\usepackage{hyperref}
% \usepackage{biblatex}
% \addbibresource{main.bib}
% \usepackage{apacite}
\usepackage{longtable}
% \usepackage[backend=biber,style=apa,citestyle=apa,sorting=ynt]{biblatex}
\usepackage{usebib} 
\bibinput{main}
\usepackage{indentfirst}
\graphicspath{ {./images/} }
\title{TUGAS }


\begin{document}
% \thispagestyle{empty}
% \begbin{center}
%     \textbf{Teknologi Informasi Lanjut}\\
%     Tugas\\
%     % Dosen: Dewi Agushinta, Dr
%     \vspace*{11\baselineskip}
%     \includegraphics[width=0.4\textwidth,height=0.4\textwidth]{glogo}  \\
%     \vspace*{11\baselineskip}
%     disusun boleh:\\
%     \textbf{Ilman Samhabib (91122010)}\\
%     \textbf{Universitas Gunadarma}\\
%     \textbf{2023}\\
% \end{center}
% \newpage

\begin{center}
  \textbf{Ujian Take Home}\\
  \textbf{Sistem Jaringan Bisnis Internasional}\\
  % Dosen: Dewi Agushinta, Dr\\
  \textbf{Ilman Samhabib (91122010)}\\
\end{center}
\section*{Soal 1}
Secara singkat (maksimal 250 kata), tuliskan analisis Anda terhadap pergerakan 25 negara
dengan aktivitas perdangan (internasional) tertinggi dari tahun 1978 - 2020 
(\url{https://unctad.org/topic/trade-analysis/chart-10-may-2021})
Anda dapat saja menyertakan data GDP atau data lain untuk memperjelas analisis Anda

\section*{Jawaban 1}

Ada beberapa faktor yang mengakibatkan negara-negara eksportir papan atas sukses dalam meningkatkan ekspor mereka:

\begin{enumerate}
  \item Kekuatan Manufaktur: Banyak negara, seperti China dan Jerman, memiliki basis manufaktur yang kuat, memproduksi berbagai barang mulai dari elektronik hingga mobil. Kemampuan ini memungkinkan mereka memenuhi permintaan global dengan harga bersaing.
  \item Keunggulan Teknologi: Negara-negara seperti AS dan Jepang China dan Korea unggul dalam inovasi, terus-menerus memajukan sektor teknologi seperti kedirgantaraan dan farmasi. Hal ini menghasilkan produk ekspor bernilai tinggi dan reputasi kualitas.
  \item Lokasi Strategis: Negara seperti Belanda dan Singapura memanfaatkan keunggulan geografis mereka. Fasilitas pelabuhan yang luar biasa, jaringan logistik, dan kedekatan dengan pasar utama memperlancar perdagangan dan meningkatkan volume ekspor. 
  \item Tenaga Kerja Terampil: Tenaga kerja yang terdidik dan terlatih sangat penting bagi negara seperti Korea Selatan dan Swiss. Keahlian ini menghasilkan produksi yang efisien, kontrol kualitas, dan keunggulan kompetitif.
  \item Dukungan Pemerintah: Banyak negara, seperti India dan Brazil, menerapkan kebijakan pendukung seperti pembiayaan ekspor, pengembangan infrastruktur, dan hibah penelitian. Hal ini menciptakan lingkungan yang kondusif bagi industri berorientasi ekspor. 
  \item Kekayaan Sumber Daya Alam: Negara-negara seperti Rusia dan Arab Saudi diberkahi dengan sumber daya alam yang melimpah. Minyak, mineral, dan produk pertanian menjadi primadona ekspor.
  \item Diversifikasi: Negara seperti Thailand dan Malaysia tidak hanya mengandalkan satu sektor. Mereka mendiversifikasi ekspor mereka di berbagai bidang seperti mobil, elektronik, makanan, dan banyak lagi, mengurangi risiko dan beradaptasi dengan perubahan permintaan pasar.
   

\end{enumerate}
\noindent
Referensi:\\
Merangkum google search dan generalisasi sebagian besar negara yang tampil pada grafis

\section*{Soal 2}
Sebagai seorang manajer investasi pada suatu perusahaan di Indonesia, Anda ditugaskan
memilih partner Asing/Internasional untuk tujuan perluasan/pengembangan
produksi/pemasaran produk/jasa perusahaan. Tentu saja, Anda sudah sangat memahami
produk/jasa perusahaan Anda. Jadi, dalam uraian pemilihan partner Internasional/asing
tersebut, Anda harus menyebutkan:
\begin{enumerate}
  \item Produk/Jasa Perusahaan Anda
  \item Posisi partner Asing/Internasional itu dalam Kerjasama tersebut: sebagai pemilik
  modal (bersama), sebagai agen, distribusi, fungsi pemasaran di negara tujuan?
  \item Faktor apa yang menjadi pertimbangan/alas an Anda memilih partner (partner)
  tersebut?
\end{enumerate}

\section*{Jawaban 2}

(\emph{coconut king} adalah nama perusahaan saya, punya pabrik pengolahan dan armada pendukung )

\subsubsection*{Produk/Jasa Perusahaan}

Produk/jasa perusahaan adalah produk kelapa olahan, yaitu minyak kelapa murni (virgin coconut oil) dan santan kelapa.  Berdasarkan produk/jasa yang ditawarkan, perusahaan membutuhkan pabrik untuk memproduksi minyak kelapa murni dan santan kelapa. Pabrik ini harus memiliki kapasitas produksi yang memadai untuk memenuhi permintaan pasar. Pabrik-pabrik tersebut
\begin{enumerate}
  \item Pabrik minyak kelapa murni: Pabrik ini dapat menggunakan metode ekstraksi uap atau ekstraksi dingin untuk menghasilkan minyak kelapa murni.
  \item Pabrik santan kelapa: Pabrik ini dapat menggunakan metode pemanasan atau tanpa pemanasan untuk menghasilkan santan kelapa.
  \item Perusahaan juga membutuhkan armada logistik untuk mengirimkan produknya ke pasar-pasar tujuan. Armada logistik ini harus memiliki kemampuan untuk mengirimkan produk dengan aman dan efisien.
  
\end{enumerate}

Supply kelapa dalam negri sangat melimpah, tetapi tetap membuka kemungkinan untuk partner asing lain untuk mensupply kelapa tambahan, agar dapat mengontrol bahkan memonopoli harga

\subsubsection*{Partner Negara Asing}

Berdasarkan pertimbangan pasar dan potensi keuntungan, partner negara asing untuk usaha ini adalah:
\begin{itemize}
  
  \item Minyak kelapa murni: Perusahaan distributor minyak kelapa murni di Amerika Serikat, Eropa, dan Jepang.(Nutiva)
  \item Santan kelapa: Perusahaan distributor santan kelapa di Asia Tenggara, Malaysia, Thailand, dan Vietnam.
  \item Supplier Kelapa: untuk menopang ekspansi, seperti (Iligan Coconut Industries, Inc., filipina)
\end{itemize}

\subsubsection*{Posisi Partner Asing/Internasional}

Posisi partner asing dalam kerjasama ini adalah sebagai distributor. Perusahaan asing tersebut akan bertanggung jawab untuk mendistribusikan produk perusahaan ke pasar tersebut.

Posisi partner asing dalam kerjasama ini juga adalah sebagai supplier jika supply dalam negri kurang.

\subsubsection*{Faktor Pertimbangan}

Faktor-faktor yang menjadi pertimbangan/alasan untuk memilih partner asing tersebut adalah:
\begin{itemize}
  
  \item Pasar: Pasar Amerika Serikat, Eropa, dan Jepang memiliki permintaan yang tinggi akan minyak kelapa murni. Pasar Asia Tenggara memiliki permintaan yang tinggi akan santan kelapa.
  \item Potensi keuntungan: Pasar Amerika Serikat, Eropa, dan Jepang memiliki potensi keuntungan yang tinggi. Pasar Asia Tenggara memiliki potensi keuntungan yang tinggi, tetapi persaingannya juga lebih ketat.
  \item Kapabilitas: Perusahaan asing tersebut memiliki kapabilitas untuk mendistribusikan produk perusahaan secara efektif dan efisien.
\end{itemize}

\noindent
Dirangkum dari berbagai sumber:\\
https://www.marketresearchfuture.com/reports/virgin-coconut-oil-market-4130\\
https://www.linkedin.com/pulse/virgin-coconut-oil-market-future-trends-latest-business\\
https://www.grandviewresearch.com/industry-analysis/virgin-coconut-oil-market-report\\
https://www.fortunebusinessinsights.com/coconut-milk-market-105439\\
https://www.fortunebusinessinsights.com/virgin-coconut-oil-market-106554\\
\url{https://icrs.gcg.gov.ph/profiles/ilicoco/?sector=Agriculture,%20Fisheries%20and%20Food%20Sector&keyword=}


\section*{Soal 3}
Temukanlah contoh sebuah Internasional Joint venture (IJV) paling berhasil sepanjang
sejarah perdagangan/bisnis dunia.
Mengapa IJV tersebut dikategorikan berhasil (Karena besarnya laba yang dihasilkan? Karena
ketenaran produk/jasa? Karena cakupan wilayah produksi/pemasarannya? Karena
keunggulan (teknologi) produk/jasa yang ditawarkan?)
Tuliskan jawaban Anda secara singkat, tidak lebih dari 500 kata.

\section*{Jawaban 3}

Di sini akan diangkat tentang, Kolaborasi Internasional Joint Venture dalam Real Estat\\

Dunia real estat semakin terhubung, dengan perusahaan dari berbagai negara dengan perbedaan keahlian dan pengetahuan unik mereka (pengalaman pada negara masing-masing) semakin sering bekerja sama untuk mengembangkan proyek dan bisnis

Berikut adalah beberapa studi kasus JV (Joint Venture) internasional pada real estat:
\begin{itemize}
  \item Tishman Speyer Properties (TSP) adalah JV antara Tishman Speyer, sebuah perusahaan real estat Amerika, dan GIC, sebuah lembaga investasi milik pemerintah Singapura. TSP telah terlibat dalam berbagai proyek real estat high-profile di seluruh dunia, termasuk revitalisasi Rockefeller Center di New York City.
  \item  Brookfield Property Partners (BPY) adalah JV antara Brookfield Asset Management, sebuah manajer aset alternatif Kanada, dan QIA, sebuah dana kekayaan negara Qatar. BPY telah terlibat dalam sejumlah proyek real estat yang sukses, termasuk akuisisi Canary Wharf Group di London.
  \item (Beda Negara) Ascendas-Singbridge Group (ASB) adalah JV antara Mitsui Fudosan, sebuah perusahaan real estat Jepang, dan CapitaLand, sebuah perusahaan real estat yang berbasis di Singapura. ASB telah mengembangkan berbagai proyek real estat perkotaan terpadu di seluruh Asia, termasuk Sino-Singapore Guangzhou Knowledge City dan International Tech Park Bangalore.
  
\end{itemize}

\subsubsection*{Faktor-faktor Keberhasilan}

Studi kasus ini menunjukkan bahwa JV internasional dapat berhasil dalam real estat. Namun, keberhasilan JV tergantung pada berbagai faktor, termasuk:
\begin{itemize}
  
  \item Sinergi: JV harus didasarkan pada kombinasi kekuatan dan sumber daya yang saling melengkapi dari perusahaan yang terlibat.
  \item Akses modal: JV harus memiliki akses ke modal yang cukup untuk mendukung proyek mereka.
  \item Pengetahuan lokal: JV harus memiliki pemahaman yang baik tentang pasar lokal tempat mereka beroperasi.
  \item Pembagian risiko: JV harus dapat membagi risiko antar perusahaan.
  \item Diversifikasi: JV harus mendiversifikasi portofolio mereka untuk mengurangi risiko.
\end{itemize}
\noindent
Referensi:\\
“Tishman Speyer Properties.” Encyclopedia.com \\
“Brookfield Property Partners L.P.” Encyclopedia.com \\
“Mitsui Fudosan Co., Ltd.” Encyclopedia.com \\
“CapitaLand Limited.” Encyclopedia.com \\
“International Joint Ventures in Real Estate.” Journal of Real Estate Literature \\


\printbibliography[title=Daftar Pustaka]




\end{document}