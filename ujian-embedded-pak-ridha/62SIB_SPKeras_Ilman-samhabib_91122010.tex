\documentclass{article}

\usepackage[a4paper,left=2.5cm,right=2.5cm,top=2.5cm,bottom=2.5cm]{geometry}
\begin{document}

\begin{center}
    Soal Ujian Sistem Perangkat Keras kelas 62-SIB\\
    Dosen: Ridha Iskandar, Dr.\\
    \smallbreak
    Ilman Samhabib -- 91122010\\
    62/MMSI/SIB  \\    
\end{center}

% Dikumpulkan ke alamat email : ridha@staff.gunadarma.ac.id dan ridhaiskandar@gmail.com
% ridha@staff.gunadarma.ac.id,
% ridhaiskandar@gmail.com
% dengan format : 62SIB_SPKeras_nama_npm_26102023.pdf

\section*{SOAL 1}
01. Apa yang Saudara ketahui mengenai pengertian CPU dan mikroprosesor?\\
\smallbreak
Semua CPU adalah microprocessor, tetapi tidak semua mikroprosessor adalah cpu

CPU biasanya bisa melakukan banyak macam tugas (instruksi instruksi di dalam cpu biasanya dapat diterapkan ke berbagai macam jenis pekerjaan/ diprogram untuk memecahkan berbagai macam kasus)

Sedangkan mikroprosesor mempunyai instruksi yang tidak terlalu beragam dan biasanya instruksi instruksi tersebut hanya didedikasikan untuk satu jenis tugas,

Tetapi didalam perkembangannya keduah istilah ini sering dianggap sama, tetapi untuk integrated ic ataupun sebuah embedded system board istilah mikroprocessor sering adalah istilah yang tepat untuk dimana pusat perhitungan dlakukan, sedangkan pada komputer ataupun PC, dan server CPU lebih sering digunakan untuk merujuk pada penghitung utama ataup pemroses utama dari semua intruksi yang ada



\section*{SOAL 2}
02.	Cobalah bandingkan 2 atau 3 sistem operasi berdasarkan pengalaman Saudara menggunakan
sistem-sistem operasi tersebut !\\
\smallbreak

\textbf{Windows dan Linux-family (biasanya Debian/Ubuntu)}\\
\textbf{Windows} menjadikan laptop atau pc yang digunakan siap pakai untuk berbagai macam bisnis casual ataupun keperluan sekolah, tetapi di balik itu semua windows adalah closed source di mana seorang pengguna yang savy (power user) mungkin tidak dapat melakukan kostumisasi lebih dalam untuk OS tersebut, dan Windows juga memiliki biaya lisensi atau penggunaannya, dengan adanya closed source juga pengguna windows mungkin tidak mengetahu sepenuhnya apa yang dilakukan komputernya, seperti update tanpa sepengetahuan user dan hal-hal lain yang dilakukan tanpa seizin pengguna, semua ini timbul karena OS windows membuat penggunanya terikat pada vendor tertentu dalam hal ini microsoft.

Sedangkan \textbf{Linux} adalah Sistem operasi yang dapat dikostumisasi seepenuh nya oleh pengguna karena sifat nya terbuka atau kode pembangunnya dapat dicek bagaimana bekerja dan kendali sepenuh nya ada ditangan pengguna, memang agak memakan waktu untuk memasang berbagai driver ataupun program agar menjadi sepraktis windows os, tetapi apa yang dilakukan Linux OS adalah hal yang dilakukan karena memang perlu, dan kalupun tidak tepat user dapat menggantinya (walau harus belajar bahasa pemrograman dan struktur OS), dan linux menganust sistem copyleft, dimana jika OS dimodifikasi dan didistribusikan ulang OS ini harus tetap bisa transparan dalam hal pembacaan source code dan semacamnya, dan linux juga tidak terikat oleh vendor tertentu karena semuanya di kembangkan dan dapat diawasi oleh publik yang berkepentingan


\section*{SOAL 3}
03.	Jabarkan sebuah sistem tertanam (embedded) yang sederhana yang Saudara kuasai secara detil !\\
\smallbreak
Kulkas Pintar/Smart Refigerator yang menggunakan kecerdasan buatan (AI) adalah jenis sistem embedded yang dapat memantau dan mengendalikan suhu, kelembaban, dan kondisi lain di dalam kulkas. Berikut adalah beberapa cara penggunaan AI olehnya:
\begin{itemize}
    \item Algoritma AI dapat digunakan untuk mengendalikan kompresor, kipas, dan komponen lainnya pada kulkas guna mengoptimalkan efisiensi energi dan menjaga suhu yang konsisten
    \item Pemeliharaan yang bersifat prediktif: AI dapat digunakan untuk memantau kinerja kulkas dan memprediksi kapan pemeliharaan atau perbaikan akan diperlukan.
    \item Manajemen makanan: AI dapat membantu pengguna mengelola persediaan makanan mereka dengan melacak tanggal kadaluarsa, memberi saran resep berdasarkan bahan yang tersedia, dan bahkan memesan bahan makanan secara otomatis
\end{itemize}

\section*{SOAL 4}
04. Apa manfaat yang bisa diambil jika sistem IoT telah sepenuhnya dijalankan ?\\
\smallbreak
Jika Iot diterapkan dengan baik ada beberapa benefit yang akan sangat terasa :

\begin{itemize}
    \item Wawasan Data secara Realtime (realtime data insight), ini adalah yang terpenting. Perangkat IoT dapat memberikan data dalam waktu nyata mengenai semua proses organisasi atau proyek, memungkinkan bisnis untuk membuat keputusan berdasarkan data dan merespons dengan cepat terhadap perubahan kondisi. Point ini adalah yang terpenting karena point point setelah nya dapat diwujudkan karena berlakunya point pertama ini
    \item  Biaya operasional yang lebih rendah: Perangkat IoT dapat membantu bisnis mengoptimalkan alur kerja dan mengurangi biaya operasional dengan memberikan informasi dalam waktu nyata, memberikan nasihat proaktif tentang status perangkat, dan terintegrasi ke dalam sistem yang lebih besar untuk mengoptimalkan efisiensi operasional.
    \item Peningkatan efisiensi dan produktivitas: Perangkat IoT dapat membantu bisnis meningkatkan produktivitas dan efisiensi dengan mengotomatisasi proses, mengurangi waktu henti, dan meningkatkan kualitas pengolahan dan analisis data besar.
    \item Peningkatan pelayanan terhadap pelanggan: Perangkat IoT dapat membantu bisnis meningkatkan pelayanandengan menyediakan layanan yang dipersonalisasi, meningkatkan kualitas produk, dan mengurangi waktu tunggu, seperti menggunakan drone pengantar dengan adanya sistem embedded di dalamnya .
    \item Lebih banyak peluang bisnis: Perangkat IoT dapat membantu bisnis mengidentifikasi sumber pendapatan baru, mengembangkan aplikasi baru untuk pengguna akhir, dan meningkatkan kemampuan mereka untuk berinovasi.
    \item Peningkatan keamanan dan kepatuhan: Perangkat IoT dapat membantu bisnis meningkatkan keamanan dan kepatuhan dengan memantau dan memprediksi anomali, mencegah kecelakaan, dan memastikan kepatuhan terhadap regulasi.
    \item  Mobilitas dan fleksibilitas yang lebih tinggi: Perangkat IoT dapat membantu bisnis menjadi lebih mobile dan fleksibel dengan memungkinkan pemantauan dan pengendalian jarak jauh terhadap perangkat, mengurangi kebutuhan untuk personil di lokasi, dan meningkatkan operasi rantai pasokan.
\end{itemize}


\section*{SOAL 5}
05.	Jelaskan layanan IaaS, PaaS dan Saas dalam sistem Clouds !\\
\smallbreak
Semua adalah jenis pelayanan yang diberikan oleh penyedia layanan IT, biasanya melalui fasilitas Cloud/internet, ketiganya berbeda dalam hal kebebasan dan kostumisasi yang diberikan kepada pengguna/konsumen nya, 
Infrastruktur sebagai Layanan (IaaS), Platform sebagai Layanan (PaaS), dan Perangkat Lunak sebagai Layanan (SaaS) Berikut adalah perbedaan antara ketiga model ini dan contoh dari masing-masing:
\begin{enumerate}
    \item Infrastruktur sebagai Layanan (IaaS):
    \begin{itemize}
        \item IaaS menyediakan akses ke sumber daya komputasi seperti mesin virtual, ataupun physical device langsung, penyimpanan, dan jaringan melalui internet.
        \item Pelanggan IaaS memiliki lebih banyak kendali atas infrastruktur dan dapat mengkonfigurasi serta mengelola mesin virtual dan penyimpanan mereka sendiri.
        \item Contoh dari IaaS termasuk AWS menggunakan EC , Microsoft Azure, dan Google Cloud Platform, ataupun linode. Semuanya memungkinkan pengguna menyewa sumber daya komputasi yang dpat digunakan untuk apapun oleh penggunanya
    \end{itemize}
    
    
    \item  Platform sebagai Layanan (PaaS):
    \begin{itemize}
        \item PaaS menyediakan platform bagi pengembang untuk membangun, menguji, dan mendeploy aplikasi melalui internet
        \item Pelanggan PaaS memiliki kendali yang lebih sedikit atas infrastruktur dan dapat fokus pada pengembangan dan penyediaan aplikasi mereka.
        \item Contoh dari PaaS termasuk Heroku, Google App Engine, dan Microsoft Azure App Service. pengguna tidak ditawarkan berapa kekuatan processor ataupun hardware tetapi tempat untuk mendeploy aplikasi mereka, dan kekuatan server biasanya sudah di diskrit kan ke dalam paket paket layanan yang menyembunyikan detail spek hardware
    \end{itemize}
    
    
    \item  Perangkat Lunak sebagai Layanan (SaaS):
     \begin{itemize}
        \item - SaaS menyediakan akses ke aplikasi perangkat lunak melalui internet, yang dihost dan dikelola oleh penyedia pihak ketiga.
        \item Pelanggan SaaS memiliki kendali yang paling sedikit atas infrastruktur dan hanya dapat menggunakan perangkat lunak yang disediakan oleh penyedia.
        \item Contoh dari SaaS termasuk Salesforce, Dropbox, dan Google Workspace, ataupun microsoft office 365 versi cloud. Pengguna biasanya langsung ditawarkan pembayaran yang memunkinkannya menggunakan perangkat lunak tertentu.
     \end{itemize}
    

    
    

 IaaS menyediakan akses ke sumber daya komputasi, PaaS menyediakan platform bagi pengembang untuk membangun dan mendeploy aplikasi, dan SaaS menyediakan akses ke aplikasi perangkat lunak. Tingkat kendali dan fleksibilitas bervariasi antara ketiga model ini, dengan IaaS menawarkan kendali paling banyak dan SaaS menawarkan kendali paling sedikit.

\end{enumerate}
\section*{SOAL 6}
06. Jabarkan tugas yang Saudara kerjakan !\\

Bab 4 mengenai Interrupt

\smallbreak
\textbf{**Poin-poin Kunci**}
\begin{itemize}
     
    \item Interrupt dalam sistem tertanam digunakan untuk mengubah alur pekerjaan sehingga dapat  memberikan respons cepat dan penanganan kejadian yang efisien. 
    \item Interrupt bisa bersifat edge-sensitive (interupt pada setiap pergantian sinyal) atau level-sensitive (interrupt hanya ketika sinyal berada pada level tertenty), masing-masing dengan kelebihan dan kekurangannya.
    
    \item ISR adalah subrutin khusus yang dipanggil saat terjadi interrupt, seperti interrupt perangkat keras atau perangkat lunak. Pertimbangan desain termasuk penggunaan tumpukan yang efisien, mengurangi laten interrupt, menghindari deadlock, dan mengelola data dengan buffer dan semafora. ISR juga harus dihindari agar tidak macet untuk mencegah prosesor terhenti.
    \item Merancang Rutinitas Pelayanan Interrupt (ISRs) sangat penting untuk sistem real-time, dan pertimbangan seperti penggunaan tumpukan (stack usage), laten interrupt, dan pencegahan deadlock harus dipertimbangkan.
    \item Mekanisme yang tepat harus diimplementasikan untuk menangani masalah seperti interrupt yang macet, konflik memori bersama, dan penggunaan registernya . Selain itu, pertimbangan harus dibuat untuk interrupt perangkat keras dan potensi masalah seperti race condition dan kesalahan waktu kumulatif yang disebabkan delay pada pengeksekusian interrupt yang mungkin datang bersamaan.
    
\end{itemize}    

\end{document}

