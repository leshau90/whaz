\documentclass{article}

\usepackage[a4paper,margin=2cm]{geometry}
\usepackage[bahasa]{babel}

\usepackage{lipsum}
\usepackage{graphicx}
\usepackage{hyperref}
% \usepackage{biblatex}
% \addbibresource{main.bib}
\usepackage{apacite}
\usepackage{longtable}
% \usepackage[backend=biber,style=apa,citestyle=apa,sorting=ynt]{biblatex}
% \addbibresource{main.bib}
\usepackage{usebib}
\usepackage{indentfirst}
\bibinput{main}
\graphicspath{ {./images/} }
\title{Tugas UJIAN: SIM dan Perencanaan Strategis }



\begin{document}

\maketitle
% \titlepage
% \newpage
% \tableofcontents
% \newpage

\section*{Soal 1}
Saat ini, Artificial Intelligence (AI) dengan mesin-mesin canggih/pintarnya telah mulai menggantikan pekerjaan manusia di perusahaan/organisasi, baik produksi maupun jasa diberbagai belahan dunia. Kondisi ini membuat kerja keras, dan jam kerja, yang panjang sampai tengah malam dari sumber daya manusia (SDM) di perusahaan/organisasi sebelumnya telah menjadi tidak banyak guna lagi. Oleh karena itu, perubahan mindset atau thinking outside the box tentang pekerjaan SDM di masa mendatang ini, apakah dapat dianggap sebagai masalah Manajemen SDM (MSDM) di perusahaan/organisasi di masa kini, khususnya SDM yang merasa terancam dengan kecanggihan AI?

Berdasarkan hal telah yang dikemukakan, maka langkah-langkah seperti apakah yang seharusnya dilakukan perusahaan/organisasi berbasis digital untuk melakukan Pengembangan SDM (PSDM) di era AI dengan keterampilan SDM yang mampu melakukan time
management didalam menjalankan aktivitasnya kreatif inovatifnya menurut pendekatan Sistem Informasi Manajemen (SIM)? (jelaskan dengan singkat dan jelas, serta gunakan asumsi yang relevan)
\subsection*{Jawaban}
AI memiliki potensi untuk menjadi baik bantuan maupun ancaman bagi karyawan dan perusahaan, tergantung pada perspektif dan implementasinya. Sebagai \textbf{bantuan}, AI dapat meningkatkan efisiensi operasional, mengotomatisasi tugas rutin, dan memberikan wawasan berharga untuk pengambilan keputusan yang lebih baik. Hal ini memungkinkan karyawan untuk fokus pada tugas yang lebih strategis, kreatif, dan berorientasi pada nilai tambah.

Namun, AI juga dapat menjadi \textbf{ancaman} jika digunakan tanpa perencanaan yang tepat. seperti, Jika perusahaan terlalu mengutamakan otomatisasi ekstrem dan menggantikan tenaga kerja manusia secara luas, hal ini dapat mengurangi lapangan kerja dan mengabaikan aspek-aspek manusiawi yang tidak dapat digantikan oleh AI, seperti empati, kreativitas, dan inovasi.

Keberhasilan transformasi kecerdasan buatan (AI) dalam sebuah perusahaan bergantung pada keputusan pengambil keputusan dan kesediaan tenaga kerja untuk mengadopsi perubahan dan memanfaatkan AI. Dengan memprioritaskan nilai-nilai humanistik, perusahaan dapat menciptakan lingkungan yang mendorong penggunaan AI sebagai pendukung yang meningkatkan efisiensi dan inovasi, sambil tetap menghargai kontribusi tak tergantikan dari tenaga kerja manusia. Namun, jika perusahaan terlalu fokus pada keuntungan ekstrim dengan penggantian besar-besaran tenaga kerja manusia oleh AI atau robotik, ini dapat mengorbankan elemen manusiawi yang penting dalam menciptakan hubungan dengan pelanggan, inovasi, dan kesuksesan jangka panjang.

\emph{Bagaimana mempersiapkan?}... kesuksesan implementasi AI bergantung pada persiapan dan pelatihan tenaga kerja. Perusahaan harus melibatkan karyawan dalam program pelatihan AI yang komprehensif. Dimulai dengan identifikasi apa yang perlu di-AI-kan, dan merencanakan integrasinya dengan infrastruktur/SDM perusahaan yang sudah ada. Jika belum lengkap maka perlu perekrutan ataupun outsurcing tenaga kerja AI seperti Data scientist ataupun Analyst (\emph{get the right expertise}) begitupun pengadaan alat-alat pintar seperti robot, dilanjutkan dengan penerapan dalam skala kecil untuk evaluasi, dan iterasi sampai tercapai tingkat skill yang diinginkan oleh perusahaan.

Pemanfaatkan potensi teknologi AI dengan maksimal dapat meningkatkan efisiensi karyawan, sehingga mampu mengambil keputusan berdasarkan data, lebih cepat menyelesaikan pekerjaan, serta meningkatkan produktifitas, sehingga karyawan itu dapat mencapai keseimbangan yang lebih baik antara pekerjaan dan kehidupan pribadi serta berkontribusi lebih terhadap tujuan perusahaan. Dengan Asumsi Adopsi pendekatan yang lebih manusiawi dengan mengembangkan tenaga kerja yang terampil dan adaptif dalam menghadapi perubahan teknologi, perusahaan dapat meraih manfaat jangka panjang dari transformasi AI, seperti setiap pekerja akan memiliki fungsi yang lebih karena pekerjaan yang bersifat \emph{chores} dapat diberikan ke AI bahkan setiap pekerja dapat menjadi manajer dalam skala kecil, sehingga memungkinkan ekspansi perusahaan yang cepat, dengan tidak menghilangkan \emph{human touch} yang mungkin masih diperlukan dalam interaksi dengan konsumen tertentu, dan masih banyak manfaat lain yang sejenis. 


% yii\cite{al2018improving}

\section*{Soal 2}
Jelaskan secara singkat tentang: (1) SIM beserta aspek-aspek yang diperlukan secara umum, (2) prospek ekonomi dan teknologi dalam implementasi SIM, serta (3) hambatan yang sering ditemui dalam pengembangan SIM di era AI.
\subsection*{Jawaban}
\textbf{(1)}
SIM (Sistem Informasi Manajemen) adalah bagaimana mengelola (berikut perencanaan dan pengealuasian) sistem informasi sehingga dapat menjadi sebuah sistem yang bermanfaat dalam pengambilan keputusan ataupun menyokong kegiatan keseharian sebuah usaha/instansi. 

\emph{Aspek-aspek} yang diperlukan secara umum adalah bagaimana menghimpun data yang diperlukan, memprosesnya menjadikan sebuah informasi yang bernilai, jika terdapat kekurangan pada proses tersebut maka direncanakan sebuah pengadaan ataupun \emph{upgrade}. Ini tentunya memrlukan aspek-aspke penyokong seperti \emph{hardware, software} dan \emph{brainware} 

\textbf{(2)} Ada beberapa hal yang dapat dicapai khususnya dalam hal teknologi dan ekonomi dengan pengimplentasian SIM
\begin{itemize}
    \item \emph{improved decision making} dengan adanya SIM yang baik maka akan ercapai sebuah decision making dalam setiap level kepengurusan perusahaan, ini kan sangat membantu dalam hal manajemen resiko, pengamatan tren, dan perencanaan jangka pendak maupun panjang
    \item \emph{economic efficiency} dengan informasi yang dikelola dengan baik dan disokong IT yang berkualitas maka proses pengadaan \emph{supply chain} dan \emph{manufacturing} begitupun --jika ada-- pemasaran akan semakin effsien sehingga dapat meningkatkan keuntungan dan menekan biaya
\end{itemize} 
\textbf{(3)} berikut hambatannya:
\begin{itemize}
    \item Besaran Data dan pemrosesanya, AI yang efektif mengharuskan volume data yang besar, apakah instansi ataupun perusahan sudah memiliki resource yang cukup untuk menyokong keperluan tersebut, 
    \item Hukum dan legalitas, Data yang besar apalagi yang didapat melalui internet kerapkali terkena masalah dengan privasi orang lain ataupun pihak tertentu
    \item pakar, pakar ai dapat dihutung dengan jari
    \item Adopsi, sebagian pekerja pasti sudah terbiasa dengan kesehariannya, dengan adanya Transformasi AI akan terjadi sebuah disrupsi yang mengharuskan sebuah pergantian yang radikal, sebagian mungkin mau berganti sebagian lain tidak, baik itu manajer ataupun pekerja ataupun para pembuat keputusan 
\end{itemize}
 
\section*{Soal 3}
Hal esensial apakah yang Saudara peroleh selama mengikuti perkuliahan (tatap muka,
tanya jawab dan diskusi, tugas kelompok dan ujian) Mk. SIM \& Perencanaan Strategis
Sistem Informasi/Tek. Informasi di Program Magister Manajemen Sistem Informasi, PPs
Universitas Gunadarma, Jakarta?
\subsection*{Jawaban}
Bagaimana berpikir \emph{out of the box} dalam menghadapi masalah yang membutuhkan  perubahan yang radikal \emph{disruptif problem that require readical changes} khususnya dalam pemikiran dan pengerjaannya. Seperti bagaimana menghadapi AI atau mencari uang yang banyak dalam waktu singkat dsb. (SD dan SMP bisa dapat duit banyak), yang jika dipikirkan secara demikian kita dapat mendapatkan solusi yang tidak perlu melakukan \emph{effort} yang berlebihan. \emph{work smart not work hard}
% \bibliographystyle{apacite}
% \bibliography{main}
% \printbibliography


\end{document}