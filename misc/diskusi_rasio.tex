\documentclass{article}
\usepackage[a4paper,left=2.5cm,right=2.5cm,top=2.5cm,bottom=2.5cm]{geometry}
\usepackage[bahasa]{babel}

% \usepackage{lipsum}
% \usepackage{graphicx}
% \usepackage{hyperref}
\begin{document}
\begin{center}
    \textbf{TUGAS 1 Manajemen Keuangan Teknologi E-Business}\\
    62/MMSI/SIB 91122010 Ilman Samhabib
\end{center}
\section*{Analisis Rasio}
(asumsi, kolom ketiga tahun 2007 dan keenam adalah 2006 pada neraca)
\renewcommand{\arraystretch}{1.5}
\begin{center}
    \begin{tabular}{ |p{7cm}|p{3.5cm}|p{3.5cm}|}
        \hline
      keterangan & 2007 & 2006\\ 
      \hline
      current ratio  & $\frac{166400}{68400}=2.43$ &$\frac{140000}{60000}=2.33$ \\ 
      \hline
      quick ratio  & $\frac{166400-71000}{68400}=1.39$ &$\frac{140000-60000}{60000}=1.33$ \\ 
      \hline
      cash ratio  & $\frac{10400}{68400}=0.15$ &$\frac{10000}{60000}=0.16$ \\ 
      \hline
      
      \hline
      total debt to total asset  & $\frac{(68400+212400)}{408400}=0.69$ 
      &$\frac{(60000+140000)}{400000}=0.5$ \\ 
      \hline
      debt to equity  & $\frac{(68400+212400)}{120000}=2.34$ 
      &$\frac{(60000+140000)}{120000}=1.67$ \\ 
      \hline
      time interest earned ratio(EBIT/interest)  & \multicolumn{2}{|c|}{$\frac{51000}{11000}=4.64$} \\ 
      \hline
      fixed charge coverage ratio(EBIT+fixed cost/fixed cost)  & \multicolumn{2}{|c|}{$\frac{51000+38000}{51000}=1.75$} \\ 
      \hline
      debt service ratio(operation/debt service operation)  & \multicolumn{2}{|c|}{$\frac{38000}{68000+212000}=0.16$} \\ 
      \hline
      perputaran persediaan(sales/inventory)  & \multicolumn{2}{|c|}{$\frac{600000}{71000}=8.45$} \\ 
      \hline
      perputaran piutang (sales(assumed as credit)/piutang)  & \multicolumn{2}{|c|}{$\frac{600000}{50000}=4.63$} \\ 
      \hline
      
      perputaran aktiva tetap (sales/net fixed asset)  & \multicolumn{2}{|c|}{$\frac{600000}{242000}=2.48$} \\ 
      \hline
      perputaran aktiva (sales/asset)  & \multicolumn{2}{|c|}{$\frac{600000}{408400}=1.47$} \\ 
      \hline
      gross profit margin  (sales-sales cost/sales)  & \multicolumn{2}{|c|}{$\frac{(600000-511000)}{600000}=0.15$} \\ 
      \hline
      profit margin  (laba kotor/ revenue)  & \multicolumn{2}{|c|}{$\frac{89000}{600000}=0.15$} \\ 
      \hline
      net profit margin  (net rev enue/revenue)  & \multicolumn{2}{|c|}{$\frac{24000}{600000}=0.04$} \\ 
      \hline
      ROA (net revenue/asset)  & \multicolumn{2}{|c|}{$\frac{24000}{408400}=0.06$} \\ 
      \hline
      ROE (net revenue/equity + laba ditahan)  & \multicolumn{2}{|c|}{$\frac{24000}{120000+76000}=0.12$} \\ 
      \hline
      ROI (net revenue/modal saham)  & \multicolumn{2}{|c|}{$\frac{24000}{120000}=0.2$} \\ 
      \hline
      EPS (net revenue/jumlah lembar saham)  & \multicolumn{2}{|c|}{$\frac{24000}{20000}=Rp 1.2/lembar$} \\ 
      \hline
      DuPont System  & Profit Margin & $\frac{24000}{600000}=0.04$ \\
       & Asset Turnover & $\frac{600000}{408400}=1.47$ \\ 
       & RoA & $0.04\times1.47=0.06$ \\ 
       & Financing Plan & $\frac{(68400+212400)}{408400}=0.69$ \\ 
       & RoE & $\frac{0.06}{1-0.69}=0.19$ \\ 
      \hline 
    \end{tabular}
\end{center}
\renewcommand{\arraystretch}{1}
\section*{Diskusi}
\textbf{Jika anda sebagai manager  keuangan, apa yang anda lakukan untuk meningkatkan nilai perusahaan}\\
Jika nilai saham adalah sebuah indikator utama dari nilai perusahaan 
maka seorang manajer keuangan harus menitikberatkan pada strategi yang meningkatkan harga saham
Dengan menitik beratkan pada tugas utama manajer keuangan 
yaitu efisiensi arus kas dan investasi jangka panjang yang tepat.\\
efisiensi arus kas ini mengelola sumber dana yang sudah ada 
dan membelanjakannya sesuai dengan tujuan perusahaan.\\ 
Investasi berjangka yang tepat adalah mengelola uang tersebut untuk program-program yang tepat yang utamanya 
adalah meningkatkan sentimen masyarakat pada prospek perusahaan ini (perusahaan baik yang selalu melakukan usaha bisnis etis dan menghormati nilai-nilai kemasyarakatan),  
setelah mempertimbangkan juga investasi untuk 
efisiensi operasi perusahaan yang tepat contohnya 
investasi pada teknologi ataupun \emph{policy} yang tepat, agar semakin efektif penggunaan input dana dan peningkatkan kualitas output produk/layanan\\ 
Tapi semua hal tersebut harus dibarengi dengan perhitungan risiko yang baik dengan memperkirakan beragam faktor eksternal untuk waktu yang akan datang
\\\\
\textbf{Menurut anda mana yang lebih penting: mengelola aktivitas investasi / mengelola aktivitas pembiayaan / aktivitas bisnis}\\
Semuanya penting karena semua nya berkaitan erat, 
tanpa adanya tata kelola yang baik untuk aktivitas bisnis maka akan sulit 
untuk memperhitungkan aktivitas investasi yang tepat untuk jangka yang lebih panjang, Dan tanpa manajemen investasi yang tepat maka akan sulit untukmenetapkan sebuah program yang berimplikasi besar dan berdampak signifikan terhadap efisiensi aktivitas bisnis\\
\\
\\
\textbf{Mengapa dewasa ini para manajer keuangan sulit  untuk mengelola keuangan perusahaan}\\
Dengan mempertimbangkan faktor selain "tidak kompeten"-nya seorang manajer keuangan, seorang manajer keuangan se-kompeten apapun harus menghadapi sebuah dunia yang semakin tidak pasti, walau globalisasi membuat berbagai hal/sumber dan prospek output semakin dekat, juga kemajuan teknologi membuat sebuah perusahaan dapat bereaksi lebih cepat, tetapi globalisasi ini juga tanpa kita sadari akan mengaitkan kita dengan berbagai hal yang juga tidak pasti. 
Hal yang terjadi di belahan dunia yang jauh pun dapat menimbulkan sebuah \emph{ripple effect} pada perusahaan yang dikelola.   
Globalisasi singkatnya menghubungkan kita terhadap volatilitas yang lebih tinggi (banyak pihak berpengaruh yang akan mengambil keputusan dan pihak pihak yang semakin banyak ini harus tetap diperhitungkan) 
yang mana menuntut kerja lebih keras dari manajer keuangan
\\
\end{document}