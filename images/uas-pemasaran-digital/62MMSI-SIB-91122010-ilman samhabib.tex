\documentclass{article}
\usepackage[bahasa]{babel}
\usepackage[a4paper,left=2cm,right=2cm,top=2cm,bottom=2cm]{geometry}

\begin{document}
\begin{center}
    \textbf{UAS Manajemen Pemasaran Berbasis Digital} -Dosen Prof. Dr. Hamdy Hady\\
    Ilman Samhabib 91122010\\
    (mengerjakan soal genap)\\   
\end{center}
 
\noindent \textbf{2-a}. Jelaskan siapa saja yg dapat menjadi Pelaku Busiess Internatioal !\\\\
Pada saat ini, dengan adanya globalisasi dan perkembangan teknologi informasi yang pesat,  bisnis internasional dapat dilakukan oleh \textbf{setiap kalangan 
perorangan ataupun kelompok} tertentu dalam berbagai skala, kita dapat membaginya sebagai berikut:
\begin{enumerate}
    \item Korporasi multinasional, atau sebuah perusahaan yang mempunyai cabang atau yang komponen \emph{suppy chain}nya ada di berbagai negara
    \item UKM atau UMKM yang mungkin meingimpor ataupun menkspor produknya
    \item Perorangan, dengan adanya platform digital yang makin memudahkan menjual/mendapatkan jasa dengan luar negri, bahkan hanya sekedar menjadi reseller untuk barang dalam maupun luar negri
    \item Negara, dengan adanya \emph{investment fund} ataupun BUMN
\end{enumerate}
bahkan kalaupun tidak terlibat dengan negara lain dalam kegiatan ekonominya,
pihak diatas tetap memiliki andil sebagai konsumen ataupun produsen sebagai pelaku bisnis 
internasional karena dunia sudah memasuki era globalisasi 
dimana barang yang dikonsumsi atau diproduksi oleh pihak tersebut terdapat andil dari negara lain ataupun dapat 
terpengaruh nilai harganya dengan apa yang terjadi di negara lain\\\\
\textbf{2-b}. Jelaskan beberapa factor yang dapat mempengaruhi aktivitas Bisnis Intertional!\\
\\berikut hal-hal yang dapat mempengaruhi aktivitas bisnis internasional
\begin{enumerate}
    \item tingkat inflasi, yang dapat menentukan kekuatan daya beli dari sebuah 
    pihak dalam sebuah negara, 
    juga tingkat harga ekspor dan impor, 
    sehingga kita dapat merencanakan/bersiap untuk bisnis pada negara  
    yang tepat untuk melakukan ekspor ataupun ekspansi bisnis
    \item kurs valas, terutamanya dapat mempengaruhi biaya operasi juga nilai barang/jasa yang diimpor ataupun ekspor
    \item tingkat bunga, dalam sebuah bisnis besar kita harus dapat mengelola utang ataupun uang yang diinvestasikan, tingkat bunga adalah hal yang harus diperhatikan karena ekspansi yang besar dan cepat seringkali membutuhkan dana yang besar, 
    kita dapat memilih dana dari bank dari negara terentu yang lebih ringan bunganya agar dapat memiliki untung yang lebih besar
    \item tingkat produksi dan investasi
    \item tingkat employment
    \item tingkat pendapatan nasional, 
    item ini dan dua di atas harus dipertimbangkan 
    untuk dapat merencanakan ataupun bereaksi dengan cepat 
    untuk tata kelola biaya produksi ataupun investasi untuk melakukan \emph{market entry/exit}, karena item-item ini sngant mempengaruhi daya beli konsumen ataupun biaya produksi pada sebuah negara tempat berbisnis 
    \item kebijakan larangan/pembatasan pemerintah, 
    setiap negara mempunyai larangan yang harus diperhatikan agar dapat melakukan bisnis dalam ranah yang legal, juga setiap negara terkadang memiliki insentif untuk suatu jenis bisnis tertentu yang dapat dimanfaatkan untuk mendapatkan keuntungan ataupun kemudahan dalam berbisnis
\end{enumerate}

\noindent\textbf{4}. Jelaskan beberapa jenis E-Business atau E-Commerce
\begin{enumerate}
    \item B 2 B (Business to Business) yaitu 2 perusahaan yang difasilitasi platform digital tertentu 
    untuk melakukan deal dalam \emph{supply chain} produksi nya agar dapat 
    saling bekerja sama dalam meningkatkan efisiensi 
    dan menambah untung, \emph{platform} B2B biasanya 
    akan memfasilitasi transaksi finansial yang cepat, bahkan pengiriman dalam skala tertentu menggunakan satu platform, juga visibilitas dalam deal seperti pelacakan pengiriman atau pemrosesan sumber daya/komoditas tertentu.
    \item B 2 C (Business to Customer) yaitu transaksi business on-line antara produsen dan konsumen, yang marak terjadi pada platform tokopedia ataupun semacam nya, dimana \emph{official store} dengan barang yang didistribusikan dari gudang pabrik terdapat pada platform tersebut.
    \item C 2 C (Customer to Customer) yaitu transaksi business antara konsumen dengan konsumen, biasanya dalam 
    bentuk penggadaian ataupun toko personal untuk barang bekas pakai pada platform tertentu seperti contohnya e-bay
    \item C 2 B (Customer to Business) yaitu transaksi business yang 
    ditawarkan konsumen kepada produsen, biasanya dalam bentuk penawaran jasa ataupun \emph{expertise} tertentu dalam sebuah platform digital, seperti contohnya \emph{linkedin} yang merupakan tempat berkumpulnya pencari kerja menawarkan skill nya ataupun \emph{fiverr} di mana para \emph{freelancer} menawarkan skill nya yang dapat dimanfaatkan bisnis tertentu untuk direkrut ataupun dipakai dalam proses bisnisnya
    \item B 2 A (Business to Administration) yaitu transaksi busines antara produsen dengan government seperti social security, fiscal, legal documents, employment, pajak ec 
    \item C 2 A (Customer to Administration) yaitu transaksi business antara consumer dengan government seperti Pendidikan, Pajak, Security Social, item ini dan item sebelumnya dapat dicontohkan dengan pelayanan pajak satupintu secara online melalui situs yang disediakan oleh DJP(dewan jendral pajak) secara online
\end{enumerate}

\noindent\textbf{6}. Jelaskan beberapa kendala masalah keamanan dalam E –Business
\\\\beberapa masalah kemanan dalam e-business adalah sebagai berikut:
\begin{enumerate}
    \item Kerahasiaan dan Pribadi. Apakah data dirahasiakan dan tidak digunakan untuk kepentingan lainnya tanpa seizin pemilik data, oleh pihak yang tidak berkepentingan bahkan oleh penyedia platform itu sendiri.  
    \item Keabsahan Data. Sifat data yang cepat dan gampang diubah dalam e-commerce mengharuskan sebuah standarisasi tertentu untuk memastikan identitas pengirim dan penerima, nahkan juga jalur pengiriman data yang digunakan. Keabsahan data dapat tercapai dengan memastikan integritas yang telah tersebut sebelumnya
    \item Integritas Data. Setelah memastikan keabsahan pengirim dan penerima maka integritas data harus dipastikan, ini dengan melakukan beberapa usaha agar data yang tersimpan tidak dapat diubah atau diganti-ganti oleh hal apapun, kecuali ketika data diperlukan untuk diganti.
    \item Tanpa Penyangkalan. Masalah yang harus dipecahkan disini adalah bagaimana agar tidak terjadi penyangkalan oleh pihak yang terlibat transaksi melakukan penyangkalan mengenai data transaksi yang melibatkan dirinya.
    \item Kontrol Akses, Masalah yang harus dipecahkan disini adalah bagaimana agar tidak terjadi akses yang dapat memodifikasi data oleh pihak yang tidak berwenang
    \item Ketersediaan Layanan, yang mana layanan harus handal demi memenuhi SLA tertentu yang sudah ditentukan, biasanya layanan harus selalu online ketika dibutuhkan
\end{enumerate}

\noindent\textbf{8}. Jelaskan 4 Konsep Manajemen Pemasaran Internasional dalam Business International
\smallbreak
\noindent Konsep-konsep dibawah ini akan menentukan strategi dalam 
ataupun bauran pemasaran dan pola manajemen yang akan dilakukan 
dalam suatu bisnis yang internasional
\begin{enumerate}
    \item Ethno - Centric
    Konsep pemasaran internasional yang berorientasi kepada Domestic Market 
    (Home Country), orientasi marketing pada daerah/negara manajer biasanya adalah yang memhami tentang pemasaran lokal pada daerah itu, dengan marketing mix berupa standardization  yaitu memberi produk dan layanan berkaitan yang sama untuk daerah tersebut                             
    \item Poly - Centric  Konsep pemasaran internasional 
    yang berorientasi kepada Multi Domestic Market (Masing – masing Host Country), disini marketing mix harus melakukan adaptation dengan menggunakan manajemen yang melokal pada negara lain tersebut.
    \item Regio – Centric  Konsep pemasaran internasional 
    yang berorientasi kepada Regional Market (Misalnya Pasar ASEAN, Amerika, Afrika, Eropa, dll), di sini diperlukan manajemen yang memahami sub region belahajn dunia ini
    \item Geo – Centric   Konsep pemasaran internasional 
    yang berorientasi kepada Global Market yang 
    sudah dianggap menyatu (satu pasar global)., untuk regio dan geo marketing mix harus melakukan standarisasi (mencoba memasarkan dengan produk yang terstandar/sama pada berbagai negara) dan adaptasi (tapi tetap memperhatikan perbedaan pada negara-negara tersebut)
    
\end{enumerate}
\textbf{10}. Jelaskan perbedaan antara Foreign Direct Investment vs Foreign Portofolio Inevstment
\\\\
\textbf{Foregin Direct Investment} adalah 
investasi langsung di mana sebuah pihak 
biasanya sebuah perusahaan langsung membeli 
sebagian besar saham/akuisisi perusahaan negara lain ataupun melakukan ekspansi bisnis baru 
di negara yang berbeda, dan telibat dalam proses manajemen langsung terhadap bisnis tersebut
, ini biasanya dilakukan dengan tujuan keuntungan jangka panjang. Sebagai contoh perusahaan otomotif jepang toyota 
membangun pabrik nya di indonesia 
dan menggunakan pekerja indonesia 
dengan harapan meningkatkan efisiensi produksi yang dapat terjadi 
karena efisiensi mata rantai produksi seperti kedekatan dengan bahan baku, 
ataupun menurunya biaya produksi dan transportasi ke negara lain, 
dan lebih dekat dengan konsumen base-nya yang lumayan besar. 
Yang pada long-termnya akan memberi keuntungan yang lebih besar,
walaupun mungkin akan merugi pada beberapa tahun awal pembangunan pabrik.\\
\\Sedangkan \textbf{Foregin Portofolio Investment} adalah di mana sebuah pihak 
membeli instrumen-instrumen keuangan pada sebuah negara lain, 
seperti saham, obligasi 
dan semacamnya, untuk diversifikasi tabungan keuangan dan memanfaat nilai tukar tertentu agar  meraup 
keuntungan jangka pendek atau jangka waktu yang sudah ditentukan sesuai 
dengan volatilitas pasar. Sebagai contoh 
seorang investor ataupun sebuah institusi finansial membeli beberapa saham pada perusahaan luar negri yang dianggap profitable ataupun beberapa obligasi pada negara lain yang mungkin dianggap baik pertumbuhan ke depannya. Walau mungkin produk seperti saham dapat dipengang dalam waktu lama sehingga akan menjadi \emph{long term investment} tapi investasi seperti ini biasanya berakhir dengan dijualnya saham ketika prospek pasar buruk dalam waktu dekat ataupun mungkin dalm waktu sedikit lebih lama.



\end{document}