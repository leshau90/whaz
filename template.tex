\documentclass{article}

\usepackage[a4paper,left=2.5cm,right=2.5cm,top=2.5cm,bottom=2.5cm]{geometry}
\usepackage[bahasa]{babel}

\usepackage{lipsum}
\usepackage{graphicx}
\usepackage{hyperref}
% \usepackage{biblatex}
% \addbibresource{main.bib}
% \usepackage{apacite}
% \usepackage{longtable}
% \usepackage[backend=biber,style=apa,citestyle=apa,sorting=ynt]{biblatex}
% \addbibresource{main.bib}
% \usepackage{usebib}
% \bibinput{main}
% \graphicspath{ {./images/} }
\title{Tugas Review Jurnal: Penggunaan SEM dalam menentukan faktor yang berkaitan dengan Sistem Informasi}
% \author{Ilman Samhabib 91122010\\62/MMSI/SIB}
% \date{\today}

\begin{document}

% \maketitle
% \titlepage
% \newpage
% \tableofcontents
% \newpage
\section{Analisa}
\subsection{Artikel 1}
\begin{tabular}{|p{2cm}|p{12cm}|}
    \hline
    Judul & E-commerce website service quality and customer loyalty
    using WebQual 4.0 with importance performances analysis,
    and structural equation model: An empirical study in Shopee\cite{ECommerceWebsWijaya2021}\\
    \hline
    Terjemahan & 
    Kualitas pelayanan e-commerce website dan loyalitas konsumen
    menggunakan Webqual 4.0 dengan IPA \emph{importance performances analysis} dan SEM: sebuah study empiris tentang \emph{shopee}\\
    % \hline
    % Pengarang & Wijaya, I Gusti Ngurah Satria; Triandini, Evi; Kabnani, Ezra Tifanie\\
    \hline
    doi & 10.26594/register.v7i2.2266\\    
    \hline
    Jurnal & Jurnal Ilmiah Teknologi Sistem Informasi\\
    % \hline
    % tahun & 2021\\    
    \hline
    Akreditasi & Sinta 1\\
    \hline
    % Link & 
    % \href{https://journal.unipdu.ac.id/index.php/register/article/viewFile/2266/pdf}{link ke artikel}\\
    % \hline
\end{tabular}

Perkembangan layanan Toko Online 
membuat penyedia layanan 
toko online harus dapat mengidentifikasi 
faktor apa yang dapat meningkatkan 
kepuasan konsumen akan layanan website dari  penyedia tersebut, 
karena ini dapat menjadi sebuah \emph{competitive edge} 
untuk manajemen dalam persaingan dengan penyedia layanan toko online lainnya

Penelitian ini  mendeskripsikan (dalam sebuah model) 
pengaruh (hubungan) qualitas website  
pada loyalitas pelanggan dengan 
menggunakan kepuasan akan layanan website e-commerce  sebagai variabel moderasinya, serta  menginvestigasi apa attribut dari \emph{webqual 4.0} 
yang mempengaruhi hubungan model ini. 
Penelitian ini mengambil objek 
website e-commerce \emph{shopee} 
dengan 104 responden konsumen \emph{shopee} 
sebagai respondennya.
Data yang dikumpulkan dianalisa 
dengan menggunakan metode 
Webqual-IPA (\emph{Importance Performance Analysis}) 
dan diverifikasi faktor-faktornya dengan SEM.

Hasil dari penelitian ini adalah 
kualitas layanan secara signifikan mempengaruhi  variable
kepuasan konsumen akan layanan website e-commerce, 
tapi kepuasan tersebut 
tidak berefek signifikan terhadap variabel loyalitas konsumen.
Serta kualitas layanan secara signifikan mempengaruhi  loyalitas konsumen.

Penelitian pun memberi rekomendasi terhadap  manajemen e-commerce
agar lebih memperhatikan attribut-attribut webqual 4.0 ini,
untuk dapat meningkatkan kepuasan dan loyalitas konsumen:\newline
 - Desain yang selaras dengan tipe website\newline
 - Pemberian informasi yang gampang untuk dimengerti\newline
 - Rasa aman dalam bertransaksi\newline
 - Pemberian Pengalaman yang positif\\
 - Pemberian informasi yang mendetail\newline

 \subsection{Artikel 2}

\smallbreak
\begin{tabular}{|p{2cm}|p{12cm}|}
    \hline
    Judul & The influence of familiarity and personal innovativeness on
    the acceptance of fintech lending services: A perspective from
    Indonesian borrowers \cite{TheInfluenceOWirani2021} \\
    \hline
    Terjemahan & Pengaruh familiaritas dan prakarsa personal 
    terhadap adopsi layanan peminjaman \emph{fintech}: Sebuah prespektif dari peminjam dari Indonesia \\
    % \hline
    % Pengarang & Wirani, Yekti; Randi, Randi; Romadhon, Muh Syaiful; Suhendi, Suhendi\\
    \hline
    doi & 10.26594/register.v8i1.2327\\    
    \hline
    Jurnal & Jurnal Ilmiah Teknologi Sistem Informasi\\
    % \hline
    % Tahun & 2022\\
    \hline
    Akreditasi & Sinta 1\\
    \hline
    % Link & 
    % \href{http://www.journal.unipdu.ac.id/index.php/register/article/view/2327}{link ke artikel}\\
    % \hline
\end{tabular}
\smallbreak
% \section*{guidlines}
% A clear and concise summary of the main findings and conclusions of the article, including key results and statistics.

% A critical evaluation of the research methods and data analysis used in the study, including the design, sample, instruments, and statistical techniques.

% An assessment of the article's contribution to the field, including its strengths and limitations, and its relevance to current research and practice.

% A discussion of the implications of the study's results for future research and practice, including any recommendations for further study or practical applications.

% A list of any relevant references cited in the article, and a review of how the article relates to previous research in the field.

% A proper formatting and clear structure, including an introduction, body, and conclusion.

% Fintech \emph{financial technology} 
% adalah sebuah kemajuan yang inovatif bagi masyarakat Indonesia, 
% karena  menawarkan berbagai 
% jasa seperti peminjaman, 
% otomasi manajemen finansial, 
% dan kemudahan bertransaksi lainnya,
% semuanya dapat dilakuakan dengan mudah
% dengan menggunakan \emph{smartphone} ataupun dengan gatget lainnya.
% Jasa peminjaman adalah sebuah aspek layanan  yang sering dimanfaatkan 
% oleh pengguna layanan Fintech di Indonesia. 
% Layanan ini kerap kali menawarkan limit dana dalam jumlah besar 
% dengan persyaratan mudah  yang tuntunya disertai dengan 
% kemudahan dalam mencairkan dana  tersebut, 
% walau terkadang bunga yang diberikan mungkin tidak rasional  
% ditengah himpitan ekonomi yang mendera banyak lapisan msyarakata. 
% Hal ini pun yang juga sering dimanfaatkan oleh pihak yang tidak bertanggung jawab, yang berkedok sebagai Perusahaan pelayanan Fintech. 
% Terbukti pada tahun 2020 pemerintah mendapatkan 126 penyedia layanan Fintech beroperasi 
% secara ilegal.

Penelitian ini berusaha menemukan faktor faktor 
yang mempengaruhi seorang pengguna di Indonesia menggunakan layanan peminjaman online. 
Beberapa faktor yang ditinjau adalah,
kepercayaan pada peneyedia layanan 
dan tingkat keamanan yang dimiliki penyedia layanan \emph{fintech}, 
prakarsa personal (\emph{personal innovativeness}), 
tingkat  bunga 
dan kemudahan.

Penelitian ini melibatkan 85 responden yang merupakan peminjam dana yang 
kebanyakan dari mereka berusia 20 sampai 25 tahun. 
Data dianalisis menggunakan CB-SEM, 
dan menemukan bahwa  
faktor prakarsa personal dan kemudahan penggunaan 
adalah faktor yang paling signifikan dalam mempengaruhi 
tingkat penerimaan pengguna terhadap penyedia layanan \emph{fintech}.


Penelitian ini berharap untuk memperjelas pengaruh beberapa faktor 
yang diharapkan dapat meningkatkan tingkat 
\emph{financial inclusiveness} (keadilan akses layanan finansial) di Indonesia. 
Seperti diantaranya, penyedia layanan dapat meningkatkan 
tingkat adopsi terhadap layananya dengan 
berusaha mempermudah pemahaman 
untuk para pengguna tentang layanan yang disediakan. 
Faktor lainnya yang mendukung adalah bahwa tingkat prakarsa personal 
yaitu bawha pengguna sendiri mempunyai rasa ingin tahu 
untuk menggunakan fasilitas kemudahan pinjaman.

Faktor lainya sperti tingkat bunga juga berpengaruh tapi tidak sesignifikan dua faktor yang disebutkan sebelumnya,
ini bisa disebabkan pengambilan sampel yang kurang mewakili populasi. 
Seperti yang dijelaskan peneliti dalam keterbatasan penelitiannya, peneliti berharap agar
sampel nantinya dapat juga mewakili masyarakat dengan demografis yang lebih luas, 
seperti pemilik usaha, ataupun responden dengan usia yang lebih bervariasi
Dengan adanya kelompok pemilik usaha sebagai responden peneliti 
berharap hasil penelitian ini dapat dikaitkan dengan model lainnya
 seperti model TOE (\emph{Technology Organization Environment}),untuk mendapatkan hasil yang lebih signifikan.



\subsection{Artikel 3}

\begin{tabular}{|p{2cm}|p{12cm}|}
    \hline
    Judul & What drives someone to share their knowledge? 
    Indonesia virtual community case\cite{WhatsDriveSomSatria2021}\\
    \hline
    Terjemahan & Apa yang membuat seseorang berbagi pengetahuannya? studi kasus komunitas virtual indonesia\\
    % \hline
    % Pengarang & Satria, Deki\\ 
    \hline
    doi & 10.26594/register.v7i2.2142\\    
    \hline
    Jurnal & Jurnal Ilmiah Teknologi Sistem Informasi\\
    % \hline
    % Tahun & 2021\\
    \hline
    Akreditasi & Sinta 1\\
    \hline
    % Link & 
    % \href{http://www.journal.unipdu.ac.id/index.php/register/article/view/2142}{link ke artikel}\\
    % \hline
\end{tabular}
Banyaknya pengetahuan dan  permasalahan yang dibagikan secara online sering kali bersifat suka rela.
Penelitian ini mencoba mengobservasi 
apa yang menjadi penyebab tingkah laku ini.
% karena mengetahui dapat membantu sebuah organisasi
%  dalam mendefinisikan sebuah fungsionalitas sistem \emph{knowledge sharing} atau dalam kata lain 
%  mempermudah pembentukan komunitas virtual dalam sebuah organisasi.
 Peneliti pun menelaah banyak artikel untuk menemukan variabel-variabel ini dapat dilakukan dengan lebih cepat karena  menggunakan cara
 PRISMA (\emph{preferred Reporting Items for Systematic
 Reviews and Meta-analyses})
 SLR(\emph{systematic literature review}).
 Artikel-artikel ini berasal dari jurnal pada IEEEXplore, Sciencedirect, and Proquest.
Berdasarkan hasil telaah peneliti, Variabel-variable yang didapatkan terdiri dari\\
- Membantu yang lain\\
- Rasa berkomunitas\\
- Mendapatkan keuntungan\\
- \emph{self efficacy}, kepastian dalam diri untuk melakukan suatu hal
- Berdiskusi\\

Peneliti kemudian mengumpulkan data menggunakan google form, 
yang menghasilkan data dari 120 responden. 
Google form ini berisi survey pertanyaan untuk 
menemukan indikasi  variabel variabel 
yang telah disimpulkan sebelumnya. 
Hasil observasi ini diproses menggunakan 
PLS-SEM untuk melakukan \emph{hypothesis testing} 
untuk menemukan mana faktor yang paling berpengaruh. 
Hasil dari pemodelan PLS-SEM ini adalah bahwa variabel 
yang mempengaruhi
secara signifikan 
kemauan seseorang untuk berbagi dalam dunia virtual adalah 
rasa berkomunitas, 
kemauan untuk membantu sesama, dan \emph{self efficacy}
\newpage
\section{Perbandingan}
\begin{longtable}{|p{1.75cm}|p{4.5cm}|p{4.5cm}|p{4.5cm}|}
\hline    
artikel 
& \cite{ECommerceWebsWijaya2021}
& \cite{TheInfluenceOWirani2021}
& \cite{WhatsDriveSomSatria2021} \\
\hline \endfirsthead
judul 
& \usebibentry{ECommerceWebsWijaya2021}{title} 
& \usebibentry{TheInfluenceOWirani2021}{title}
& \usebibentry{WhatsDriveSomSatria2021}{title} \\
\hline

SEM
&CB-SEM (covariance based)
&PLS-SEM (Partial Least Square)
&PLS-SEM \\
\hline


\end{longtable}



% PLS
% Partial Least Squares (PLS) is a regression technique used to analyze the relationship between a set of predictor variables and a set of response variables. The goal of PLS is to find a set of latent variables that explain the covariance between the predictor and response variables, and to use these latent variables to make predictions about the response variables based on the predictor variables.

% PLS is typically performed in the following steps:

%     Data preparation: The data is prepared by centering and scaling the predictor and response variables.

%     Calculation of covariance matrix: The covariance matrix is calculated between the predictor and response variables.

%     Calculation of latent variables: The latent variables are calculated using singular value decomposition (SVD) on the covariance matrix. The latent variables are orthogonal and ordered by their contribution to explaining the covariance between the predictor and response variables.

%     Regression analysis: Regression analysis is performed using the latent variables as the independent variables and the response variables as the dependent variables. The regression coefficients are used to make predictions about the response variables based on the latent variables.

%     Validation of the model: The model is validated by comparing the predictions to the actual values of the response variables. The model can be improved by adjusting the number of latent variables used in the analysis.

%     Model interpretation: The results of the PLS analysis can be used to interpret the relationship between the predictor and response variables and to make predictions about the response variables based on the predictor variables.

% Overall, PLS is a useful method for analyzing the relationship between a set of predictor and response variables, and can provide a more robust solution in cases where the data is not normally distributed or has high levels of multicollinearity.
% SEM
% Structural Equation Modeling (SEM) is a statistical method that can be used after performing Partial Least Squares (PLS) analysis to build a more comprehensive model of the relationships between variables.

% PLS is a regression technique that can be used to analyze the relationship between a set of predictor variables and a set of response variables by finding a set of latent variables that explain the covariance between the predictor and response variables.

% In SEM, the results of the PLS analysis are used as the input to build a structural equation model. The structural equation model is a set of equations that describe the relationships between the variables in a more comprehensive manner, taking into account not only the direct relationships between the predictor and response variables, but also the indirect relationships between variables that are mediated by other variables.

% The structural equation model is estimated using regression techniques and can be used to make predictions about the outcome variables based on the predictor variables. The SEM model can also be used to perform hypothesis testing and to assess the fit of the model to the data.

% In the case of PLS Structural Equation Modeling (PLS-SEM), the regression coefficients estimated in the PLS analysis are used as the input to the SEM model, providing a more robust and flexible solution to the traditional SEM methods.

% Overall, SEM provides a more comprehensive method for analyzing the relationships between variables, and PLS-SEM combines the strengths of PLS and SEM to provide a robust and flexible solution for analyzing complex relationships between multiple predictor and multiple outcome variables.

% \tableofcontents
% \newpage
% \section{manageable}
% \begin{enumerate}
%     \item Detecting violent scenes in movies 
%     using Gated Recurrent Units 
%     and Discrete Wavelet Transform 
%     \item Effect of information gain on document 
%     classification using k-nearest neighbor,\\ 
%     without information gain the model yields better accuracy
%     \item Spatial dynamics model of land use and land cover changes:
%     A comparison of CA, ANN, and ANN-CA. \\
%     menentukan faktor \emph{driver} dan melakukan forecasting
%     \item Fuzzy-AHP MOORA approach for vendor selection
%     applications, memilih vendor switch
%     \item Use of online applications in maintaining MSMEs
%     performance during the COVID-19 pandemic, using wilcoxon test to see if there a significant difference between the shop that uses online shopping 
%     and the shop that don't, 
%     an the finding says it is significantly different
%     \item Simulation of TOPSIS calculation in Discrepancy-Tat Twam
%     Asi evaluation model. The mysterious logic of TOPSIS is based on the concept 
%     that the chosen alternative should have the shortest geometric 
%     distance from the best solution 
%     and the longest geometric distance 
%     from the worst solution. 
%     Such methodology allows finding trade-offs between 
%     criteria when a poor performance in one can be canceled by a good performance in another criterion. This provides a pretty comprehensive form of modeling because we are not excluding alternative solutions based on pre-defined thresholds.
%     \item Land-use  suitability  
%     evaluation for organic  rice  
%     cultivation usingfuzzy-AHP ELECTRE method, demi mengurangi tungkat degradasi tanah karena over farming, maka digunakan fuzzy bla bla
%     untuk mencari daerah yang tepat untuk melakukan 
%     \item  Analysis of e-learning 
%     readiness level of public 
%     and private universities 
%     in Central Java, Indonesia., finding out the readiness level of university,  using carefully crafted questionares
    
    

%     \item TOPSIS for mobile based group and personal decision support
% system, adding BORDA to TOPSIS and 
% testing its correlation aginst just TOPSIS the
%     \item Identifying Degree-of-Concern on COVID-19 topics with text
%     classification of Twitters, embeddin tiwtter text with word2ved or fasttext and CNN,RNN,LSTM to classify if the corpus concern 
%     on covid 19 is true 
%     \item Movie recommender systems using hybrid model based on graphs with co-rated, 
%     genre, and closed caption features

% \end{enumerate}
% \newpage
% \bibliographystyle{apacite}
% \bibliography{main}
% \printbibliography


\end{document}