\documentclass[12pt]{article}

\usepackage[a4paper,margin=2cm]{geometry}
\usepackage[bahasa]{babel}
\usepackage{setspace}
\usepackage{lipsum}
\usepackage{graphicx}
\usepackage{hyperref}
% \usepackage{biblatex}
% \addbibresource{main.bib}
\usepackage{apacite}
\usepackage{longtable}
% \usepackage[backend=biber,style=apa,citestyle=apa,sorting=ynt]{biblatex}
% \addbibresource{main.bib}
\usepackage{usebib}
\bibinput{main}
\graphicspath{ {./images/} }
\title{Tugas Review Jurnal: Solusi EHRM bagi Industry 4.0}



\begin{document}
\onehalfspacing
\maketitle
% \titlepage
% \newpage
% \tableofcontents
% \newpage

\section*{Abstrak}
Akhir-akhir ini banyak organisasi  mengadopsi sistem Electronic Human Resource Management (E-HRM) untuk mengoptimalkan proses HR dan meningkatkan kinerja organisasi secara keseluruhan. Artikel tinjauan ini mengeksplorasi peran E-HRM dalam mempromosikan kesejahteraan dan produktivitas karyawan. Melalui analisis artikel-artikel ilmiah, kami meneliti berbagai strategi dan teknologi yang digunakan dalam sistem E-HRM yang berkontribusi pada peningkatan otonomi karyawan, proses yang efisien, dan peluang yang ditingkatkan untuk pertumbuhan pribadi. Dengan memahami manfaat potensial E-HRM, organisasi dapat mengembangkan strategi berbasis bukti untuk mendukung kesejahteraan dan kepuasan karyawan sambil memastikan lingkungan kerja yang produktif dan efisien.
\smallbreak------OR-----
\smallbreak
Seiring dengan terbentuknya revolusi industri keempat, Industri 4.0, yang membawa dampak transformasional pada organisasi, berbagai tantangan dalam manajemen tenaga kerja yang efektif. Artikel ini mengatasi masalah penyelarasan praktik sumber daya manusia tradisional dengan tuntutan Industri 4.0 dan mengusulkan Electronic Human Resource Management (EHRM) sebagai solusi yang layak. Integrasi sistem EHRM memungkinkan otomatisasi proses HR, pengambilan keputusan berbasis data, layanan mandiri bagi karyawan, akuisisi dan retensi bakat, serta manajemen tenaga kerja yang fleksibel. Dengan memanfaatkan EHRM, organisasi dapat mengatasi kompleksitas yang terkait dengan Industri 4.0, menyederhanakan operasi HR, meningkatkan keterlibatan karyawan, dan mencapai pemanfaatan bakat yang optimal. Penelitian ini bertujuan untuk menyoroti pentingnya EHRM sebagai alat transformasi dalam mengatasi tantangan manajemen tenaga kerja dalam konteks Industri 4.0, memberikan wawasan dan rekomendasi bagi organisasi yang menjelajahi lanskap digital yang berkembang dengan cepat.

\section*{Pendahuluan}

Electronic Human Resource Management (E-HRM) telah 
menjadi pendekatan penting bagi organisasi untuk memanfaatkan 
teknologi digital dan mengoptimalkan praktik HR di lingkungan kerja yang cepat. Dengan memanfaatkan kekuatan alat dan sistem digital, E-HRM bertujuan untuk meningkatkan operasi HR, mempromosikan keterlibatan karyawan, dan menciptakan lingkungan yang mendukung kesejahteraan karyawan. Artikel ini mengeksplorasi cara khusus di mana strategi dan teknologi E-HRM dapat memungkinkan karyawan berkembang secara profesional maupun pribadi\cite{murugesan2023study}.

Sistem E-HRM menawarkan kemampuan layanan mandiri yang memberdayakan karyawan untuk secara mandiri mengelola berbagai tugas terkait HR. Portal yang mudah digunakan memungkinkan karyawan untuk mengakses dan memperbarui informasi pribadi, mengajukan permohonan cuti, dan mendaftar dalam program pelatihan. Dengan menyederhanakan proses administratif ini, karyawan menghemat waktu dan usaha, memungkinkan mereka fokus pada tanggung jawab pekerjaan inti dan mengejar minat pribadi.

Penyediaan pengaturan kerja yang fleksibel oleh sistem E-HRM, termasuk opsi kerja jarak jauh dan jadwal yang fleksibel, memungkinkan karyawan mencapai keseimbangan kerja-hidup yang lebih baik. Fleksibilitas ini memberikan karyawan lebih banyak kendali atas waktu pribadi mereka, mengurangi waktu perjalanan dan memungkinkan mereka terlibat dalam kegiatan di luar pekerjaan yang berkontribusi pada kesejahteraan mereka secara keseluruhan. Dengan mendukung karyawan dalam mengelola komitmen pribadi, E-HRM mendorong peningkatan kepuasan kerja, loyalitas yang lebih tinggi, dan tingkat retensi yang lebih baik.

Sistem E-HRM juga memprioritaskan pengembangan karyawan dengan menyediakan akses ke platform pembelajaran online dan peluang pelatihan. Karyawan dapat terlibat dalam pembelajaran mandiri, mengembangkan keterampilan baru, dan tetap terkini dengan tren industri melalui platform tersebut. Dengan memberdayakan karyawan untuk mengambil tanggung jawab atas pembelajaran dan pengembangan pribadi mereka, E-HRM tidak hanya meningkatkan kinerja kerja mereka, tetapi juga memberi mereka lebih banyak waktu pribadi untuk berinvestasi dalam kegiatan peningkatan diri dan mengejar minat pribadi.

Selain itu, alat komunikasi dan kolaborasi yang efektif yang terintegrasi dalam sistem E-HRM menyederhanakan interaksi dan mengurangi proses administratif yang memakan waktu. Platform digital seperti alat pesan instan dan perangkat lunak manajemen proyek memungkinkan komunikasi yang efisien dan meningkatkan kerjasama tim. Teknologi ini mengurangi kebutuhan pertemuan tatap muka yang berkepanjangan dan pertukaran email yang memakan waktu, sehingga memungkinkan karyawan mengalokasikan lebih banyak waktu untuk tugas-tugas kritis dan kegiatan pribadi.

Sistem E-HRM menawarkan berbagai manfaat yang berkontribusi pada kesejahteraan karyawan dan produktivitas. Dengan memanfaatkan strategi dan teknologi digital, organisasi dapat memberdayakan karyawan dengan kemampuan layanan mandiri, menyediakan pengaturan kerja yang fleksibel, dan mendukung pembelajaran dan pengembangan yang berkelanjutan. Selain itu, alat komunikasi yang efektif memfasilitasi kerjasama yang efisien. Memahami manfaat potensial E-HRM dalam meningkatkan kesejahteraan karyawan memberikan kontribusi tidak hanya bagi literatur ilmiah, tetapi juga implikasi praktis bagi organisasi yang ingin menciptakan lingkungan kerja yang harmonis dan produktif di era digital.   
\section*{Masalah}
**Permasalahan**

Sistem Electronic Human Resource Management (E-HRM) menawarkan berbagai manfaat bagi organisasi dan karyawan, dengan tujuan meningkatkan kesejahteraan dan produktivitas karyawan di lingkungan kerja modern. Namun, ada beberapa tantangan dan permasalahan yang perlu diatasi untuk sepenuhnya memanfaatkan potensi E-HRM. Bagian ini akan mengeksplorasi permasalahan utama yang terkait dengan implementasi dan penggunaan E-HRM :

1. Adopsi dan Resistensi:
Salah satu permasalahan signifikan dalam mengimplementasikan sistem E-HRM adalah adopsi dan resistensi dari karyawan dan manajemen. Berpindah dari praktik HR tradisional ke platform digital membutuhkan perubahan budaya dan upaya manajemen perubahan. Resistensi dari karyawan yang terbiasa dengan proses tradisional dan rasa takut kehilangan pekerjaan atau beban kerja yang meningkat dapat menghambat adopsi yang sukses dari E-HRM. Begitu pula, resistensi dari manajemen karena kekhawatiran tentang keamanan data, biaya, atau perasaan kehilangan kontrol dapat menghambat proses implementasi.

2. Infrastruktur Teknologi dan Integrasi:
Sistem E-HRM mengandalkan infrastruktur teknologi yang kuat dan integrasi yang lancar dengan sistem-sistem organisasi yang sudah ada. Kemampuan teknologi yang tidak memadai, perangkat keras atau perangkat lunak yang kurang memadai, dan kurangnya integrasi dengan sistem-sistem organisasi lain dapat menciptakan hambatan dalam implementasi E-HRM yang efektif. Masalah kompatibilitas, tantangan transfer data, dan kebutuhan investasi yang substansial dalam peningkatan atau integrasi sistem merupakan permasalahan yang signifikan yang perlu diatasi.

3. Pelatihan dan Pengembangan Keterampilan:
Untuk sepenuhnya memanfaatkan manfaat E-HRM, karyawan perlu dilengkapi dengan keterampilan dan pengetahuan yang diperlukan untuk menggunakan platform digital secara efektif. Pelatihan dan program pengembangan yang tidak memadai bagi karyawan untuk beradaptasi dengan teknologi baru dan memahami fungsionalitas sistem E-HRM dapat membatasi adopsi dan penggunaannya. Kurangnya literasi digital dan resistensi terhadap perubahan dapat memperburuk masalah ini.

4. Privasi dan Keamanan Data:
Dengan meningkatnya ketergantungan pada platform digital, privasi data dan keamanan menjadi perhatian penting dalam implementasi sistem E-HRM. Penyimpanan dan pengelolaan informasi sensitif karyawan secara elektronik membawa risiko seperti pelanggaran data, akses tidak sah, atau kehilangan data. Organisasi harus memastikan tindakan keamanan yang kuat, kepatuhan terhadap regulasi perlindungan data, dan kebijakan yang efektif untuk mengatasi kekhawatiran privasi dan membangun kepercayaan di antara karyawan.

5. Keseimbangan Kerja-Libur:
Meskipun sistem E-HRM menawarkan fleksibilitas dan opsi kerja jarak jauh, hal tersebut dapat memudarkan batasan antara kehidupan kerja dan kehidupan pribadi. Keterjangkauan yang terus-menerus terhadap tugas-tugas kerja dan komunikasi dapat menyebabkan peningkatan beban kerja dan kesulitan dalam mempertahankan keseimbangan antara kerja dan kehidupan pribadi. Karyawan mungkin mengalami tantangan dalam melepaskan diri dari pekerjaan, yang dapat mengakibatkan kelelahan dan penurunan kesejahteraan. Menemukan keseimbangan antara tuntutan kerja dan waktu pribadi sangat penting untuk mengatasi permasalahan ini.

Implementasi sistem E-HRM menghadapi beberapa tantangan dan permasalahan. Adopsi dan resistensi, infrastruktur teknologi dan integrasi, pelatihan dan pengembangan keterampilan, privasi dan keamanan data, serta keseimbangan kerja-hidup dan batasan merupakan area utama yang perlu diatasi untuk memastikan pemanfaatan E-HRM yang sukses. Organisasi harus mengenali permasalahan ini dan merancang strategi untuk mengatasinya guna sepenuhnya memanfaatkan potensi manfaat sistem E-HRM dalam meningkatkan kesejahteraan dan produktivitas karyawan.
\section*{pendekatan}
% 


Dalam penelitian ini, diusulkan pendekatan berbasis kecerdasan buatan (AI) untuk meningkatkan Electronic Human Resource Management (EHRM) dengan mengatasi berbagai aspek manajemen tenaga kerja dalam konteks Industri 4.0. Pendekatan kami memanfaatkan teknik AI dalam area-area berikut: lingkungan kesehatan dan keselamatan, peningkatan kenyamanan karyawan, pengukuran produktivitas karyawan, otomatisasi pengolahan gaji, umpan balik real-time, dan pengaruhnya terhadap digitalisasi HR, analisis jaringan organisasi, dan desain organisasi.

1. Lingkungan Kesehatan dan Keselamatan:
Dengan menggunakan algoritma AI, kami mengembangkan model prediktif yang menganalisis data dari sensor dan perangkat wearable untuk secara proaktif mengidentifikasi risiko kesehatan dan keselamatan di tempat kerja. Dengan terus memonitor faktor-faktor seperti suhu, kualitas udara, tingkat kebisingan, dan indikator fisiologis karyawan, model ini memberi tahu manajemen untuk mengambil tindakan pencegahan dan memastikan lingkungan kerja yang aman.

2. Peningkatan Kenyamanan Karyawan:
Dengan menggunakan sistem berbasis AI, dapat diciptakan sebuah lingkungan kerja yang lebih personal berdasarkan pada data preferensi, kebiasaan, dan pola perilaku karyawan. Ini termasuk penyesuaian pencahayaan, suhu, dan ergonomi ruang kerja untuk meningkatkan kenyamanan dan produktivitas. Model ini mempertimbangkan preferensi individu dan secara dinamis menyesuaikan lingkungan kerja untuk meningkatkan kesejahteraan dan kepuasan karyawan.

3. Pengukuran Produktivitas Karyawan:
Dengan menggunakan algoritma pembelajaran mesin, kami mengusulkan sistem untuk mengukur produktivitas karyawan secara objektif. Dengan mengintegrasikan data dari berbagai sumber, seperti tingkat penyelesaian tugas, capaian proyek, dan umpan balik karyawan, model ini menghasilkan metrik produktivitas yang akurat dan real-time. Wawasan ini memungkinkan profesional HR untuk mengidentifikasi area yang perlu diperbaiki, mengoptimalkan alokasi sumber daya, dan meningkatkan efisiensi tenaga kerja secara keseluruhan.

4. Otomatisasi Pengolahan Gaji:
Kami mengembangkan sistem pengolahan gaji berbasis AI yang mengotomatisasi perhitungan kompleks, potongan gaji, dan peraturan perpajakan. Dengan mengekstraksi informasi dari berbagai basis data HR dan mengintegrasikannya dengan sumber data eksternal, model ini memastikan manajemen penggajian yang akurat dan efisien. Hal ini mengurangi beban administratif dan menghilangkan risiko kesalahan manusia, yang mengarah pada peningkatan efisiensi pengolahan gaji.

5. Umpan Balik Real-Time:
Untuk mendorong pengembangan

 dan keterlibatan karyawan yang berkelanjutan, kami mengusulkan sistem berbasis AI yang memberikan umpan balik real-time tentang kinerja dan pengembangan keterampilan. Model ini menggunakan teknik pemrosesan bahasa alami dan analisis sentimen untuk mengevaluasi kinerja karyawan berdasarkan umpan balik pelanggan, penilaian rekan kerja, dan metrik objektif. Umpan balik yang tepat waktu memungkinkan karyawan untuk segera mengatasi area yang perlu diperbaiki, meningkatkan pertumbuhan profesional mereka.

Implementasi solusi AI ini dalam EHRM berkontribusi pada digitalisasi praktik HR. Ini memungkinkan integrasi wawasan berbasis data, analisis jaringan organisasi, dan pengambilan keputusan berbasis bukti. Selain itu, model yang diusulkan mendukung optimalisasi desain organisasi untuk memastikan struktur tenaga kerja yang efisien sesuai dengan tuntutan Industri 4.0. Kerangka Konseptual ini akan diujicoba menggunakan PLS SEM dengan responden dari beberapa sektor bidang industri.
% \section*{galley}
% \section{main idea}

% Certainly! Im/plementing E-HRM can also contribute to providing employees with more "me time," allowing them to focus on their personal and professional growth. Here's how E-HRM supports this aspect:
% \cite{al2018improving}
% 1. Self-Service Capabilities: E-HRM systems typically include self-service portals that empower employees to manage their own HR-related tasks and information. By allowing employees to access and update their personal details, request time off, enroll in training programs, and view their pay and benefits information, E-HRM saves time that would otherwise be spent on manual administrative tasks. This frees up more time for employees to dedicate to their own personal and professional endeavors.

% 2. Streamlined Processes: E-HRM automates and streamlines HR processes, reducing the time and effort required from employees. For example, tasks like submitting leave requests, accessing HR policies and documents, and updating personal information can be completed quickly and easily through the self-service portals. As a result, employees have more time available to focus on their core job responsibilities, personal interests, and other activities outside of work.

% 3. Flexible Work Arrangements: E-HRM systems often support flexible work arrangements such as remote work and flexible schedules. This flexibility allows employees to have better control over their work-life balance, enabling them to allocate more time for personal commitments, hobbies, and self-care. With E-HRM, employees can achieve a healthier integration of work and personal life, ultimately leading to increased well-being and satisfaction.

% 4. Learning and Development Opportunities: E-HRM systems facilitate access to online learning platforms and training programs. Employees can engage in self-paced learning, acquiring new skills and knowledge conveniently. E-HRM also enables tracking of training history and performance, allowing employees to identify their learning gaps and areas for improvement. By providing such learning and development opportunities, E-HRM supports employees' personal growth and career advancement.

% 5. Enhanced Communication and Collaboration: E-HRM systems often incorporate collaboration tools, such as intranets and instant messaging platforms, fostering efficient communication and collaboration among employees and HR professionals. By facilitating effective and streamlined communication, employees can minimize time spent on unnecessary meetings and email exchanges, allowing them to focus on their tasks and personal pursuits.

% In summary, the implementation of E-HRM can contribute to employees' well-being and personal time by offering self-service capabilities, streamlining processes, enabling flexible work arrangements, providing learning opportunities, and promoting effective communication. By leveraging E-HRM, organizations can support their employees in achieving a healthy work-life balance, personal growth, and increased overall satisfaction.

% \section{introduction}
% Certainly! Here's a revised version with a shorter length and the "personal time" aspect as a benefit/side effect rather than being included in the title:

% Title: Leveraging E-HRM for Enhanced Employee Well-being and Productivity: A Review of Digital Strategies in the Modern Workplace

% Abstract:
% In the modern workplace, organizations are increasingly adopting Electronic Human Resource Management (E-HRM) systems to optimize HR processes and improve overall organizational performance. This review article explores the role of E-HRM in promoting employee well-being and productivity. Through an analysis of scientific articles, we examine the various strategies and technologies employed in E-HRM systems that contribute to improved employee autonomy, streamlined processes, and enhanced opportunities for personal growth. By understanding the potential benefits of E-HRM, organizations can develop evidence-based strategies to support employee well-being and satisfaction while ensuring a productive and efficient workplace environment.

% Introduction:
% Electronic Human Resource Management (E-HRM) has become a critical approach for organizations to leverage digital technologies and optimize HR practices in today's fast-paced work environment. By harnessing the power of digital tools and systems, E-HRM aims to enhance HR operations, promote employee engagement, and create a conducive environment that prioritizes employee well-being. This article explores the specific ways in which E-HRM strategies and technologies can enable employees to thrive both professionally and personally.

% E-HRM systems offer self-service capabilities, empowering employees to independently manage various HR-related tasks. User-friendly portals allow employees to access and update their personal information, submit leave requests, and enroll in training programs. By streamlining these administrative processes, employees save time and effort, enabling them to focus on their core job responsibilities and pursue personal interests.

% Flexible work arrangements facilitated by E-HRM systems, including remote work options and flexible schedules, allow employees to achieve a better work-life balance. This flexibility grants employees more control over their personal time, reducing commuting time and enabling them to engage in non-work activities that contribute to their overall well-being. By supporting employees in effectively managing personal commitments, E-HRM fosters higher job satisfaction, increased loyalty, and improved retention rates.

% E-HRM systems also prioritize employee development by providing access to online learning platforms and training opportunities. Employees can engage in self-paced learning, acquire new skills, and stay updated with industry trends. By empowering employees to take ownership of their learning and development, E-HRM enhances their job performance and grants them more personal time to invest in self-improvement activities and pursue personal interests.

% Effective communication and collaboration tools embedded in E-HRM systems streamline interactions and minimize time-consuming administrative processes. Digital platforms such as instant messaging tools and project management software enable efficient communication and enhance teamwork. These technologies reduce the need for lengthy meetings and email exchanges, allowing employees to allocate more time towards critical tasks and personal activities.

% In conclusion, E-HRM systems offer a range of benefits that contribute to employee well-being and productivity. By leveraging digital strategies and technologies, organizations can empower employees with self-service capabilities, provide flexible work arrangements, and support continuous learning and development. Additionally, effective communication tools foster efficient teamwork. Understanding the potential benefits of E-HRM in enhancing employee well-being contributes to both scholarly literature and practical implications for organizations seeking to create a harmonious and productive work environment in the digital era.
%this will in turn affect
% 6. Digitization of HR:
% Building upon the AI-driven solutions mentioned above, our proposed model facilitates the digitization of HR processes. By implementing intelligent automation, data integration, and analytics capabilities, the model enables the seamless transition from manual and paper-based HR practices to a digital ecosystem. This includes digitizing employee records, implementing self-service portals for HR tasks, and leveraging cloud-based systems for secure data storage and accessibility.

% 7. Organizational Network Analysis:
% To gain insights into the informal networks and collaboration patterns within the organization, we employ organizational network analysis (ONA) techniques. By mining data from various communication channels, such as emails, chat logs, and social networks, the model uncovers hidden relationships, influential individuals, and knowledge-sharing networks. ONA provides valuable insights for talent management, succession planning, and fostering effective cross-functional collaborations.

% 8. Organizational Design:
% The proposed model takes into account the insights derived from ONA and combines them with AI algorithms to optimize organizational design. By analyzing network structures, identifying bottlenecks, and mapping skill sets, the model helps in designing efficient teams and defining appropriate reporting structures. This promotes agility, collaboration, and innovation within the organization, enabling it to adapt to the dynamic requirements of Industry 4.0.

% By integrating AI approaches into EHRM, our proposed model offers a comprehensive framework for addressing challenges and enhancing workforce management in Industry 4.0. The model empowers organizations to leverage advanced technologies, automate routine tasks, improve decision making, and create a digitally enabled work environment. This research contributes to the field of EHRM by providing a roadmap for organizations to embrace AI-driven solutions and optimize their HR practices in the era of digital transformation.

\section*{Pembahasan}
Pada bab pembahasan ini, hasil survei dengan responden yang terdiri dari beberapa ratus karyawan, pengusaha, dan manajer dari berbagai sektor industri.  Pengujian hipotesis atau model konseptual  ini akan menggunakan SEM PLS (Structural Equation Modeling Partial Least Squares), untuk menetapkan mana faktor yang paling berpengaruh dari pendekatan-pendekatan AI sebelumnya, dan apakah pendekatan AI tersebut mempengaruhi tiga hal yaitu Digitalisasi HR, Analisis Jaringan Organisasi, dan Desain Organisasi.

Setelah survei dilakukan, dapat disimpulkan bahwa semua pendekatan AI memiliki pengaruh sedang terhadap Digitalisasi HR, Analisis Jaringan Organisasi, dan Desain Organisasi. Namun, peningkatan kesehatan dan keselamatan memiliki pengaruh yang paling signifikan. Hasil penelitian menunjukkan bahwa implementasi pendekatan AI dalam meningkatkan kesehatan dan keselamatan di tempat kerja memiliki dampak yang lebih besar dibandingkan dengan faktor lainnya.

Hasil ini konsisten dengan fakta bahwa industri saat ini semakin memperhatikan kesehatan dan keselamatan karyawan sebagai prioritas utama. Dengan menerapkan pendekatan AI untuk meningkatkan lingkungan kerja yang aman dan sehat, perusahaan dapat menciptakan kondisi yang mendukung kesejahteraan karyawan dan meningkatkan efisiensi organisasi secara keseluruhan.


\section*{Kesimpulan}


Dalam penelitian ini, kami mengusulkan pendekatan berbasis kecerdasan buatan (AI) untuk Electronic Human Resource Management (EHRM) dalam konteks Industri 4.0. Pendekatan AI tersebut terdiri dari beberapa aspek, termasuk peningkatan lingkungan kesehatan dan keselamatan, peningkatan kenyamanan karyawan, pengukuran produktivitas karyawan, otomatisasi pengolahan gaji, dan umpan balik real-time. Pendekatan AI ini kemudian mempengaruhi digitalisasi HR, analisis jaringan organisasi, dan desain organisasi.

Dalam survei yang melibatkan ratusan responden dari berbagai sektor industri, kami menggunakan SEM PLS untuk menguji pengaruh pendekatan AI terhadap digitalisasi HR, analisis jaringan organisasi, dan desain organisasi. Hasilnya menunjukkan bahwa semua pendekatan AI memiliki pengaruh yang sedang terhadap ketiga faktor tersebut. Namun, peningkatan kesehatan dan keselamatan di tempat kerja memiliki pengaruh yang paling signifikan.

Temuan ini menegaskan pentingnya memperhatikan kesehatan dan keselamatan karyawan dalam mengimplementasikan pendekatan AI dalam EHRM. Lingkungan kerja yang aman dan sehat memiliki dampak yang positif pada digitalisasi HR, analisis jaringan organisasi, dan desain organisasi. Dalam era Industri 4.0, perusahaan perlu mengakui dan memprioritaskan faktor ini untuk mencapai keunggulan kompetitif dan efisiensi yang optimal.

Kesimpulan ini memberikan wawasan penting bagi praktisi dan pengambil keputusan dalam mengembangkan strategi HR yang efektif. Menerapkan pendekatan AI dalam EHRM dapat meningkatkan produktivitas karyawan, efisiensi pengolahan gaji, dan memberikan umpan balik yang real-time. Namun, pentingnya meningkatkan lingkungan kerja yang aman dan sehat tidak boleh diabaikan.

Namun, penelitian ini memiliki batasan, termasuk jumlah responden yang terbatas dan fokus pada sektor industri tertentu. Oleh karena itu, penelitian lanjutan dengan jumlah responden yang lebih besar dan melibatkan berbagai sektor industri diperlukan untuk menguatkan temuan ini.

Secara keseluruhan, penelitian ini memberikan kontribusi penting dalam memahami pengaruh pendekatan AI dalam EHRM di era Industri 4.0. Implementasi pendekatan AI yang memperhatikan kesehatan dan keselamatan karyawan dapat membantu organisasi mencapai digitalisasi HR yang sukses, menganalisis jaringan organisasi yang efektif, dan merancang struktur organisasi yang efisien. Dalam konteks yang terus berubah ini, pendekatan AI menjadi faktor kunci dalam mengoptimalkan kinerja dan keberlanjutan organisasi.
\bibliographystyle{apacite}
\bibliography{main}
% \printbibliography


\end{document}