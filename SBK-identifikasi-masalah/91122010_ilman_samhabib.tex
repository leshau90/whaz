\documentclass{article}
\usepackage[margin=2cm]{geometry}
\usepackage[bahasa]{babel}

% \usepackage{lipsum}
% \usepackage{graphicx}
% \usepackage{hyperref}
\begin{document}
\begin{center}
    \textbf{Identifikasi Masalah}\\
    Seminar Bidang Kajian
    \bigbreak
    Integrasi SSO (Single Sign On) pada Sistem-sistem Informasi Universitas X
    \\ \emph{Ilman Samhabib}
\end{center}
Di era digital saat ini,
universitas sangat bergantung
pada berbagai aplikasi berbasis web untuk mengelola operasinya,
seperti Sistem Informasi Sumberdaya Terintegrasi, Sistem Manajemen Anggaran, Sistem Informasi Akademis, Sistem Informasi Wisuda, dan sebagainya.
Setiap aplikasi tersebut memerlukan pengguna untuk
melakukan otentikasi dengan kredensial
login yang berbeda-beda,
yang dapat menjadi proses yang
rumit dan memakan waktu.
Selain itu, mengelola beberapa nama pengguna
dan kata sandi meningkatkan risiko pelanggaran
keamanan dan menimbulkan beban administratif
bagi departemen IT universitas.
Untuk mengatasi tantangan ini, integrasi Single Sign-On (SSO) dapat menjadi solusi yang efektif.

SSO memungkinkan pengguna untuk mengakses
beberapa aplikasi dengan satu set kredensial
login saja, karena banyaknya kredensial yang diurus sangat membebani pengguna.
Namun, mengimplementasikan sistem SSO dalam
infrastruktur universitas yang sudah ada dapat menjadi
tantangan karena kompleksitas
dan saling ketergantungan dari berbagai sistem yang ada,
terutama sistem-sistem yang telah digunakan dibangun dengan menggunakan bahasa pemrograman dan \emph{framework} yang berbeda-beda.
Proses integrasi memerlukan perencanaan,
pengujian, dan pelatihan yang cermat untuk
memastikan transisi yang mulus
dan meminimalkan gangguan bagi
pengguna universitas dan
untuk memastikan transisi yang lancar.


Integrasi
sistem SSO akan memudahkan pengguna
universitas untuk mengakses berbagai aplikasi.
Sistem-sistem yang akan dintegrasikan contohnya,
Sistem Informasi Sumberdaya Terintegrasi,
Sistem Manajemen Anggaran,
Sistem Informasi Akademis,
Sistem Informasi Wisuda,
dan sebagainya dengan hanya menggunakan satu
set kredensial login.
Hal ini akan mengurangi waktu
dan usaha yang dibutuhkan untuk melakukan
otentikasi serta meningkatkan keamanan dan akurasi data.

\bigbreak
beberapa ide yang mungkin akan ditambahkan.
Apakah sistem SSO diterima dengan baik,
dengan diukur menggunakan model TAM (\emph{technology acceptance model}) ataupun pengembangan-pengembangannya

Certainly! Here's the LaTeX code to generate the table I provided earlier:

\begin{table}[h!]
    \centering
    \caption{Summary of SSO Articles and Case Studies}
    \begin{tabular}{|p{4cm}|p{3cm}|p{2cm}|p{3cm}|p{3cm}|}
        \hline
        \textbf{Title of the Article / Case Study}                               & \textbf{Author's Name}                                    & \textbf{Year of Publication} & \textbf{Issue/Problem}                                                  & \textbf{Solution/Approach}                                                                                                            \\
        \hline
        "Integrating Single Sign-On into Existing Systems"                       & Khalid Al-Tahat and Khaled Alghathbar                     & 2016                         & Challenges in integrating SSO with existing systems                     & Proposed a model for integrating SSO with existing systems using a three-layer architecture                                           \\
        \hline
        "Single Sign-On Implementation using OpenID Connect"                     & Seyed Mohammad Fazeli, Alireza Darvishy, and Flavius Kehr & 2017                         & Challenges in implementing SSO using OpenID Connect                     & Proposed a model for implementing SSO using OpenID Connect and evaluated its effectiveness through experiments                        \\
        \hline
        "Implementing Single Sign-On in a Healthcare Environment"                & Karim Abdullah                                            & 2013                         & Challenges in implementing SSO in a healthcare environment              & Proposed a model for implementing SSO in a healthcare environment and evaluated its effectiveness through case study                  \\
        \hline
        "Single Sign-On Integration for a Web-Based Financial Services Platform" & Yuqiao Song and Xiaojie Yin                               & 2019                         & Challenges in integrating SSO with a financial services platform        & Proposed a model for integrating SSO with a web-based financial services platform and evaluated its effectiveness through experiments \\
        \hline
        "Integrating Single Sign-On for a Cloud-Based Collaboration Platform"    & David Roldán-Martínez et al.                              & 2017                         & Challenges in integrating SSO with a cloud-based collaboration platform & Proposed a model for integrating SSO with a cloud-based collaboration platform and evaluated its effectiveness through experiments    \\
        \hline
        "Implementing SSO in a Higher Education Environment"                     & Priyanka Singh and Abhishek Joshi                         & 2018                         & Challenges in implementing SSO in a higher education environment        & Proposed a model for implementing SSO in a higher education environment and evaluated its effectiveness through case study            \\
        \hline
    \end{tabular}
    \label{tab:sso_summary}
\end{table}




\begin{table}[h!]
    \centering
    \caption{bard version}
    \begin{tabular}{|p{2.5cm}|p{1cm}|p{5cm}|p{4cm}|}
        \hline
        Title & Year & Problem & Solution \\
        \hline
        A Survey of Single Sign-On Technologies & 2022 & The need for a single sign-on (SSO) solution is growing as more and more applications are being deployed in the cloud. However, there are a number of challenges associated with SSO, such as security, scalability, and usability. This paper surveys the existing SSO technologies and discusses the challenges and solutions associated with each technology.& \\
        \hline
        A Novel SSO Framework for Federated Identity Management & 2021 & The traditional SSO framework is based on the client-server model. However, this model has a number of limitations, such as security vulnerabilities and scalability issues. This paper proposes a novel SSO framework for federated identity management. The proposed framework is based on the distributed ledger technology and provides a number of advantages over the traditional SSO framework, such as enhanced security, scalability, and interoperability. &\\
        \hline
        A Comparative Study of SSO Protocols & 2020 & There are a number of SSO protocols available, each with its own advantages and disadvantages. This paper compares the three most popular SSO protocols: SAML, OAuth 2.0, and OpenID Connect. The paper discusses the features and security of each protocol and provides a recommendation for the best protocol for a given application. &\\
        \hline
        A Survey of Security Issues in SSO Systems & 2019 & SSO systems are increasingly being used to provide a single sign-on experience for users. However, SSO systems are also vulnerable to a number of security attacks. This paper surveys the security issues in SSO systems and discusses the security measures that can be taken to mitigate these attacks. &\\
        \hline        
    \end{tabular}
\end{table}




\end{document}