\documentclass{article}

\usepackage[a4paper,left=2.5cm,right=2.5cm,top=2.5cm,bottom=2.5cm]{geometry}
\usepackage[bahasa]{babel}

\usepackage{lipsum}
\usepackage{graphicx}
\usepackage{hyperref}
% \usepackage{biblatex}
% \addbibresource{main.bib}
\usepackage{apacite}
\usepackage{longtable}
% \usepackage[backend=biber,style=apa,citestyle=apa,sorting=ynt]{biblatex}
% \addbibresource{main.bib}
\usepackage{usebib}
\usepackage{indentfirst}
\bibinput{main}
\graphicspath{ {./images/} }
\title{}

\begin{document}
\begin{center}
    Tugas SIM dan Perencanaan Strategis SI/TI\\
    Balanced Scorecard\\
    62/MMSI/SIB\\
    Ilman Samhabib 91122010\\
    Tri Lestari 91122021\\	
\end{center}
\subsection*{Abstrak}

Balanced Scorecard (BSC) memiliki peran penting dalam perencanaan dan perusahaan dengan mengintegrasikan empat perspektif utama. Makalah ini menggambarkan pentingnya BSC dalam mengarahkan upaya perusahaan, mengukur kinerja holistik, dan mengidentifikasi peluang perbaikan. Selain itu, penelitian ini juga membahas pengembangan terbaru, termasuk pengenalan Digital Balanced Scorecard, yang secara khusus menghadapi tantangan dan peluang transformasi digital. Dengan implementasi yang efektif, BSC memungkinkan perusahaan meningkatkan keselarasan strategis, efisiensi operasional, dan keunggulan kompetitif dalam lingkungan bisnis yang dinamis.
\subsection*{Pendahuluan}
\emph{Balanced Scorecard} (BSC) memiliki peran penting dalam perencanaan dan perusahaan. Sebagai sebuah kerangka kerja yang komprehensif, BSC membantu dalam merumuskan dan mengkomunikasikan strategi perusahaan dengan mengintegrasikan empat perspektif utama: Keuangan, Pelanggan, Proses Internal, dan Pembelajaran dan Pertumbuhan. Dengan mengadopsi BSC, perusahaan dapat mengarahkan upaya mereka secara terkoordinasi, mengukur kinerja secara holistik, dan mengidentifikasi kekuatan serta peluang perbaikan. BSC juga memfasilitasi pengambilan keputusan yang informasional dengan menyediakan pandangan yang seimbang antara tujuan jangka pendek dan jangka panjang. 
\subsection*{Pembahasan}
\subsection*{Kesimpulan}


% \bibliographystyle{apacite}
% \bibliography{main}
\end{document}