\documentclass{article}

\usepackage[a4paper,left=2.5cm,right=2.5cm,top=2.5cm,bottom=2.5cm]{geometry}
\usepackage[bahasa]{babel}

\usepackage{lipsum}
\usepackage{graphicx}
\usepackage{hyperref}
% \usepackage{biblatex}
% \addbibresource{main.bib}
\usepackage{apacite}
\usepackage{longtable}
% \usepackage[backend=biber,style=apa,citestyle=apa,sorting=ynt]{biblatex}
% \addbibresource{main.bib}
\usepackage{usebib}
\bibinput{main}
\graphicspath{ {./images/} }
\title{Tugas Review/Kritik Jurnal: Programmatic Advertising}
\author{Ilman Samhabib 91122010\\62/MMSI/SIB}
\date{\today}

\begin{document}

% \maketitle
% \titlepage
% \newpage
% \tableofcontents
% \newpage
\begin{center}
    Tugas Review/Kritik Artikel Ilmiah
    \\ Tema: \textbf{Programmatic Advertising} 
    \\ 991122010 62/MMSI/SIB
    \\ Ilman Samhabib
\end{center}
\section*{Artikel}
\begin{center}
    \begin{tabular}{|p{2cm}|p{12cm}|}
        \hline
        Judul & A study of the effects of programmatic advertising on users' concerns about privacy overtime \cite{PALO1961} \\
        \hline
        Terjemahan & Studi tentang Pengaruh Pengiklanan Programatik terhadap kekhawatiran pengguna akan privasi  seiring berjalannya waktu\\
        \hline
        Pengarang & Pedro Palos-Sanchez, Jose Ramon Saura, Felix Martin-Velicia\\
        \hline
        doi & doi.org/10.1016/j.jbusres.2018.10.059\\    
        \hline
        Jurnal & Journal of Business Research\\
        % \hline
        % tahun & 2021\\    
        \hline
        % Link & 
        % \href{https://journal.unipdu.ac.id/index.php/register/article/viewFile/2266/pdf}{link ke artikel}\\
        % \hline
    \end{tabular}
    
\end{center}
---
\begin{center}
     \begin{tabular}{|p{2cm}|p{12cm}|}
        \hline
        Judul & Analisis Pengaruh Promosi Digital Dan Kualitas Layanan Aplikasi Go-Food Terhadap Keputusan Pembelian \cite{mahmud2023analisis}
         \\
        % \hline
        % Terjemahan & Pengiklanan (secara) Programatis: interpretasi kekhawatiran-kekhawatiran konsumen \\
        \hline
        Pengarang & Mahmud, Amir\\
        \hline
        doi & 10.36778/jesya.v6i1.921\\    
        \hline
        Jurnal & Jesya (Jurnal Ekonomi dan Ekonomi Syariah)\\
        % \hline
        % Tahun & 2022\\
        \hline
        % Akreditasi & Q1\\
        % \hline
        % Link & 
        % \href{http://www.journal.unipdu.ac.id/index.php/register/article/view/2327}{link ke artikel}\\
        % \hline
    \end{tabular}
\end{center}
\section*{Pengantar}

\indent Programmatic Advertising (PA) merupakan bentuk periklanan yang menggunakan teknologi profilisasi lalu lintas internet, seperti penggunaan cookies, untuk mengidentifikasi minat dan preferensi pengguna. Dalam hal ini, PA memungkinkan pengiklan untuk secara efisien menargetkan audiens yang tepat, mengoptimalkan pengeluaran iklan, dan mengurangi birokrasi yang terkait. Dengan pendekatan yang lebih terarah, pengiklan hanya membayar ketika iklan mereka dilihat, diklik, atau menghasilkan tindakan yang diinginkan, seperti pembelian produk. Hal ini memastikan bahwa dana yang diinvestasikan dalam periklanan memiliki dampak yang lebih efektif dan efisien.

Meskipun PA menawarkan manfaat yang signifikan dalam hal penghematan dana dan pengurangan birokrasi, namun terdapat kekhawatiran terkait privasi. Dalam konteks PA, pengumpulan dan penggunaan data pengguna menjadi penting. Beberapa orang mungkin merasa tidak nyaman dengan fakta bahwa data pribadi mereka dikumpulkan dan digunakan untuk menyusun profil pengguna. Kejelasan mengenai bagaimana data dikumpulkan, digunakan, dan disimpan oleh platform PA menjadi penting dalam memastikan bahwa privasi pengguna tetap terlindungi.

Dalam menghadapi kekhawatiran privasi ini, perlu dilakukan langkah-langkah yang memastikan transparansi dan kontrol pengguna atas penggunaan data mereka. Pengiklan dan platform PA harus mengikuti standar privasi yang ketat dan memberikan pilihan bagi pengguna untuk mengatur preferensi mereka terkait penggunaan data. Dengan melakukan hal ini, pengguna dapat merasa lebih aman dan nyaman dalam berinteraksi dengan periklanan berbasis PA, sementara pengiklan tetap dapat memanfaatkan teknologi ini dengan bertanggung jawab.
\section*{Penelitian dan Pendekatan}
Penelitian \cite{mahmud2023analisis} bertujuan untuk menganalisis pengaruh Layanan Go-Food sebagai media promosi dan Kualitas Layanan Go-Food terhadap keputusan pembelian makanan di Kota Makassar. \cite{mahmud2023analisis} menggunakan metode pengumpulan data, seperti uji validitas, uji reliabilitas, analisis regresi berganda, koefisien determinasi, uji F, dan uji t untuk menguji hipotesis yang diajukan.

Hasil penelitian \cite{mahmud2023analisis} menunjukkan bahwa secara simultan, promosi digital dan kualitas pelayanan Go-Food berpengaruh positif dan signifikan terhadap keputusan pembelian dengan nilai signifikan 0,015  0,05. Variabel promosi digital dan kualitas layanan secara positif dan signifikan mempengaruhi keputusan pembelian. Hasil penelitian juga menunjukkan bahwa keputusan pembelian di outlet makanan dipengaruhi oleh promosi digital dan kualitas layanan Go-Food, dengan R square sebesar 37,9, sementara sisanya dipengaruhi oleh faktor-faktor lain yang tidak diteliti dalam penelitian ini. Di sini kita dapat melihat bahwa penggunaan pengiklanan digital, yang efisien dapat meningkatkan penjualan baik secara online dan langsung ke \emph{outlet}

Penelitian \cite{PALO1961} juga mencoba menganalisis pengaruh Perceived Usefulness (Kebergunaan yang Dirasakan) dari Programmatic Advertising (PA) terhadap Concern about Privacy (Kekhawatiran Privasi) pengguna dan memeriksa apakah hubungan ini berubah seiring waktu. Dalam hal ini, tujuan penelitian adalah untuk mengetahui bagaimana peningkatan efektivitas periklanan programmatic dapat meningkatkan kekhawatiran pengguna terhadap privasi. Penelitian ini juga menginvestigasi apakah hubungan ini meningkat seiring waktu dan apakah ada efek yang berkaitan dengan waktu. 

\cite{PALO1961} menggunakan data dari sampel pengguna internet yang sangat besar di Spanyol (n = 14.822) dianalisis pada tiga periode waktu yang berbeda antara tahun 2013 dan 2017. PLS-SEM digunakan untuk analisis seperti yang banyak diterapkan dalam studi sosial. Beberapa kelompok dianalisis untuk menguji perbedaan antara koefisien jalur variabel laten pada momen-momen waktu yang berbeda. Dengan menggunakan studi longitudinal memungkinkan untuk menyelidiki tidak hanya keberadaan hubungan, tetapi juga bagaimana hubungan tersebut berubah dari tahun ke tahun. Dalam Penelitian \cite{PALO1961} mengungkapkan dalam penelitiannya adalah kurangnya kesadaran pengguna akan bagaimana penggunaan cookies atau geolokasi mempengaruhi privasi mereka, terutama dalam konteks pesan programmatic/dibuat menggunakan algoritma yang tidak diinginkan.  

\cite{PALO1961} menginvestigasi pengaruh yang dirasakan dari kegunaan programmatic advertising (PA) online terhadap kekhawatiran privasi pengguna. Tujuan penelitian ini adalah memberikan ide kepada manajer agensi periklanan dan pengembang sistem periklanan programmatic online baru, serta kepada para pemegang merek yang ingin hadir di Internet menggunakan medium periklanan ini, yang dapat bermanfaat dalam perancangan dan implementasi kampanye periklanan. Hasil penelitian menunjukkan bahwa kegunaan yang dirasakan memengaruhi kekhawatiran privasi pengguna, dan hubungan ini tetap terlihat selama lima tahun penelitian. Meskipun terdapat hubungan yang signifikan antara kegunaan yang dirasakan dan kekhawatiran privasi, tidak terdapat bukti bahwa hubungan ini semakin kuat seiring waktu. Selain itu, penelitian ini menyoroti pentingnya perlindungan privasi pengguna internet dan implikasinya dalam praktik periklanan online.

Penelitian ini juga menemukan bahwa kepentingan privasi meningkat seiring waktu, dan bahwa privasi akan menjadi faktor yang semakin penting di masa depan. Meskipun pengguna internet aktif mengkhawatirkan privasi mereka, kekhawatiran ini semakin meningkat ketika mereka menyadari bagaimana PA bekerja dan bagaimana hal tersebut mempengaruhi privasi mereka. Namun demikian, pengguna juga menganggap positif ketika mereka diperlihatkan produk dan layanan yang sesuai dengan minat mereka.

Hasil penelitian ini juga menunjukkan bahwa ada potensi persaingan antar situs web dalam upaya meningkatkan kunjungan, waktu koneksi, tingkat penolakan, atau lead, karena pengguna dapat memilih untuk menggunakan browser, mesin pencari, dan situs web yang tidak mencatat perilaku jaringan pengguna. Selain itu, penelitian ini mengindikasikan bahwa tingkat kegunaan yang dirasakan dan kekhawatiran privasi meningkat seiring waktu, meskipun tidak secara signifikan.

Penelitian ini mengakhiri dengan menyelidiki mediasi atau perubahan dalam hubungan antara variabel independen (X) dan variabel dependen (Y) sebagai akibat dari pengenalan variabel mediasi (M). Hipotesis mediasi yang diajukan tidak mendukung perubahan signifikan dalam opini yang diungkapkan pada tahun 2015, dibandingkan dengan tahun 2013 dan 2017. Hal ini mungkin disebabkan karena tidak ada kemajuan teknologi yang mempengaruhi kegunaan yang dirasakan, atau berita yang muncul pada tahun 2015 tentang kekhawatiran privasi atau regulasi perlindungan data di Eropa belum sepenuhnya dipertimbangkan oleh pengguna internet.
\section*{Penutup}

Hasil penelitian \cite{PALO1961} menunjukkan bahwa iklan di internet telah menjadi alat yang dapat mempengaruhi keinginan pengguna terhadap produk atau layanan tertentu, tetapi keputusan tersebut bukan diambil oleh mereka, melainkan oleh perusahaan periklanan yang menggunakan jenis iklan ini untuk mempersonalisasi pengalaman pengguna di internet secara massal. Penelitian ini juga menyimpulkan bahwa pengguna masih belum sepenuhnya menyadari dampak privasi dari penggunaan cookies dan geolokasi, serta kurangnya pelatihan dalam nilai-nilai etika yang terkait langsung dengan internet. Jadi walau \cite{mahmud2023analisis} menemukan bahwa memang kita dapat meningkatkan penjualan dengan biaya yang lebih efisian, tapi perlu diingat bahwa menurut \cite{PALO1961}
secara keseluruhan mengungkapkan perlunya meningkatkan kesadaran pengguna tentang privasi internet dan pentingnya regulasi yang memadai terkait praktik periklanan programmatic.
% \begin{longtable}{|p{1.75cm}|p{4.5cm}|p{4.5cm}|p{4.5cm}|}
% \hline    
% artikel 
% & \cite{ECommerceWebsWijaya2021}
% & \cite{TheInfluenceOWirani2021}
% & \cite{WhatsDriveSomSatria2021} \\
% \hline \endfirsthead
% judul 
% & \usebibentry{ECommerceWebsWijaya2021}{title} 
% & \usebibentry{TheInfluenceOWirani2021}{title}
% & \usebibentry{WhatsDriveSomSatria2021}{title} \\
% \hline

% SEM
% &CB-SEM (covariance based)
% &PLS-SEM (Partial Least Square)
% &PLS-SEM \\
% \hline


% \end{longtable}

\bibliographystyle{apacite}
\bibliography{main}
% \printbibliography


\end{document}